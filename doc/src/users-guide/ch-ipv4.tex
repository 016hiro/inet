% based on 'integration' branch: 446a1265c28822456bc0230d362bbabdee7778ba (2012-03-19)

\chapter{IPv4}
\label{cha:ipv4}


\section{Overview}

The IP protocol is the workhorse protocol of the TCP/IP protocol suite.
All UDP, TCP, ICMP packets are encapsulated into IP datagrams and
transported by the IP layer.
While higher layer protocols transfer data among two communication end-point,
the IP layer provides an hop-by-hop, unreliable and connectionless delivery
service. IP does not maintain any state information about the individual
datagrams, each datagram handled independently.

The nodes that are connected to the Internet can be either a host or a router.
The hosts can send and recieve IP datagrams, and their operating system
implements the full TCP/IP stack including the transport layer. On the
other hand, routers have more than one interface cards and perform packet
routing between the connected networks. Routers does not need the
transport layer, they work on the IP level only. The division
between routers and hosts is not strict, because if a host
have several interfaces, they can usually be configured to operate
as a router too.

Each node on the Internet has a unique IP address. IP datagrams contain
the IP address of the destination. The task of the routers is to find
out the IP address of the next hop on the local network, and forward
the packet to it. Sometimes the datagram is larger, than the maximum
datagram that can be sent on the link (e.g. Ethernet has an 1500 bytes limit.).
In this case the datagram is split into fragments and each fragment is
transmitted independently. The destination host must collect all fragments,
and assemble the datagram, before sending up the data to the transport
layer.

\subsection{INET modules}

The INET framework contains several modules to build the
IPv4 network layer of hosts and routers:
\begin{itemize}
  \item \nedtype{IPv4} is the main module that implements RFC791. This
        module performs IP encapsulation/decapsulation, fragmentation
        and assembly, and routing of IP datagrams.
  \item The \nedtype{IPv4RoutingTable} is a helper module that manages the routing
        table of the node. It is queried by the \nedtype{IPv4} module
        for best routes, and updated by the routing daemons implementing
        RIP, OSPF, Manet, etc. protocols.
  \item The \nedtype{ICMP} module can be used to generate ICMP error packets. It also
        supports ICMP echo applications.
  \item The \nedtype{ARP} module performs the dynamic translation of IP addresses
        to MAC addresses.
  \item The \nedtype{IGMPv2} module to generate and process multicast group
        membership reports.
\end{itemize}

These modules are assembled into a complete network layer module
called \nedtype{IPv4NetworkLayer}. This module has
dedicated gates for TCP, UDP, SCTP, RSVP, OSPF, Manet, and Ping
higher layer protocols. It can be connected to several network
interface cards: Ethernet, PPP, Wlan, or external interfaces.
The \nedtype{IPv4NetworkLayer} module is used to build IPv4 hosts
(\nedtype{StandardHost}) and routers (\nedtype{Router}).

The implementation of these modules are based on the following RFCs:
\begin{itemize}
  \item RFC791: Internet Protocol
  \item RFC792: Internet Control Message Protocol
  \item RFC826: Address Resolution Protocol
  \item RFC1122: Requirements for Internet Hosts - Communication Layers
  \item RFC2236: Internet Group Management Protocol, Version 2
\end{itemize}

The subsequent sections describe the IPv4 modules in detail.

\section{The IPv4 Module}

The \nedtype{IPv4} module implements the IPv4 protocol.

For connecting the upper layer protocols the \nedtype{IPv4} module
has \emph{transportIn[]} and \emph{transportOut[]} gate vectors.

The IP packets are sent to the \nedtype{ARP} module through the
\emph{queueOut} gate. The incoming IP packets are received
directly from the network interface cards through the
\emph{queueIn[]} gates. Each interface card knows its own
network layer gate index.


\subsection{IP packets}

IP datagrams start with a variable length IP header.
The minimum length of the header is 20 bytes, and
it can contain at most 40 bytes for options, so
the maximum length of the IP header is 60 bytes.

\begin{center}
\begin{bytefield}{32}
\bitheader{0,3,4,7,8,15,16,18,19,23,24,31} \\
\bitbox{4}{Version} &
\bitbox{4}{IHL} &
\bitbox{8}{\small Type of Service} &
\bitbox{16}{Total Length} \\
\bitbox{16}{Identification} &
\bitbox{3}{Flags} &
\bitbox{13}{Fragment Offset} \\
\bitbox{8}{Time to Live} &
\bitbox{8}{Protocol} &
\bitbox{16}{Header Checksum} \\
\bitbox{32}{Source Address} \\
\bitbox{32}{Destination Address} \\
\bitbox{24}{Options} &
\bitbox{8}{Padding} \\
\end{bytefield}
\end{center}

The \ttt{Version} field is 4 for IPv4. The 4-bit \ttt{IHL} field is the
number of 32-bit words in the header. It is needed because the header
may contain optional fields, so its length may vary. The minimum IP header
length is 20, the maximum length is 60. The header is always padded to
multiple of 4 bytes. The \ttt{Type of Service} field designed to store
priority and preference values of the IP packet, so applications can
request low delay, high throughput, and maximium reliability from the
routing algorithms. In reality these fields are rarely set by applications,
and the routers mostly ignore them. The \ttt{Total Length} field is the
length of the whole datagram in bytes. The \ttt{Identification} field
is used for identifying the datagram sent by a host. It is usually generated
by incrementing a counter for each outgoing datagram. When the datagram
gets fragmented by a router, its \ttt{Identification} field is kept unchanged
to the other end can collect them. In datagram fragments the \ttt{Fragment Offset}
is the address of the fragment in the payload of the original datagram. It is
measured in 8-byte units, so fragment lengths must be a multiple of 8.
Each fragment except the last one, has its \ttt{MF} (more fragments) bit set
in the \ttt{Flags} field. The other used flag in \ttt{Flags} is the \ttt{DF}
(don't fragment) bit which forbids the fragmentation of the datagram.
The \ttt{Time to Live} field is decremented by each router in the path,
and the datagram is dropped if it reached 0. Its purpose is to prevent
endless cycles if the routing tables are not properly configured, but
can be used for limiting hop count range of the datagram (e.g. for local
broadcasts, but the \fprog{traceroute} program uses this field too).
The \ttt{Protocol} field is for demultiplexing the payload of the IP
datagram to higher level protocols. Each transport protocol has a registered
protocol identifier. The \ttt{Header Checksum} field is the 16-bit one's
complement sum of the header fields considered as a sequence of 16-bit numbers.
The \ttt{Source Address} and \ttt{Destination Address} are the IPv4 addresses
of the source and destination respectively.

The \ttt{Options} field contains 0 or more IP options. It is always padded
with zeros to a 32-bit boundary. An option is either a single-byte option
code or an option code + option length followed by the actual values for
the option. Thus IP implementations can skip unknown options.

An IP datagram is represented by the \msgtype{IPv4Datagram} message class.
It contains variables corresponding the fields of the IP header, except:
\begin{itemize}
  \item \fvar{Header Checksum} omitted, modeled by error bit of packets
  \item \fvar{Options} only the following options are permitted and the
                       datagram can contain at most one option:
        \begin{itemize}
          \item Loose Source Routing
          \item Strict Source Routing
          \item Timestamp
          \item Record Route
        \end{itemize}
\end{itemize}

The \fvar{Type of Service} field is called \ttt{diffServCodePoint} in
\nedtype{IPv4Datagram}.

Before sending the \msgtype{IPv4Datagram} through the network, the \nedtype{IPv4}
module attaches a \cppclass{IPv4RoutingDecision} control info.
The control info contains the IP address of the next hop, and the
identifier of the interface it should be sent. The ARP module translate
the IP address to the hardware address on the local net of the specified
interface and forwards the datagram to the interface card.


\subsection{Parameters}

The \nedtype{IPv4} module has the following parameters:
\begin{itemize}
  \item \fpar{procDelay} processing time of each incoming datagram.
  \item \fpar{timeToLive} default TTL of unicast datagrams.
  \item \fpar{multicastTimeToLive} default TTL of multicast datagrams.
  \item \fpar{protocolMapping} string value containing the \ttt{protocol id}
        $\rightarrow$ \ttt{gate index} mapping, e.g. \ttt{``6:0,17:1,1:2,2:3,46:4''}.
  \item \fpar{fragmentTimeout} the maximum duration until fragments are kept
          in the fragment buffer.
  \item \fpar{forceBroadcast} if \fkeyword{true}, then link-local broadcast
          datagrams are sent out through each interface, if the higher
          layer did not specify the outgoing interface.
\end{itemize}

% compile time options: WITH\_MANET, NEWFRAGMENT

\subsection{Statistics}

The \nedtype{IPv4} module does not write any statistics into files,
but it has some statistical information that can be watched during
the simulation in the gui environment.
\begin{itemize}
  \item \ttt{numForwarded}: number of forwarded datagrams, i.e. sent to one of the
        interfaces (not broadcast), counted before fragmentation.
  \item \ttt{numLocalDeliver}: number of datagrams locally delivered.
        (Each fragment counted separately.)
  \item \ttt{numMulticast}: number of routed multicast datagrams.
  \item \ttt{numDropped} number of dropped packets.
        Either because there is no any interface, the interface is not specified and
        no \fpar{forceBroadcast}, or received from the network but IP forwarding disabled.
  \item \ttt{numUnroutable}: number of unroutable datagrams, i.e. there is no
        route to the destination. (But if outgoing interface is specified it is routed!)
\end{itemize}

In the graphical interface the bubble of the \nedtype{IPv4} module
also displays these counters.


\section{The IPv4RoutingTable module}

The \nedtype{IPv4RoutingTable} module represents the routing table.
IP hosts and routers contain one instance of this class. It has
methods to manage the routing table and the interface table,
so one can achieve functionality similar to the \fprog{route} and
\fprog{ifconfig} commands.

This is a simple module without gates, it requires function calls to it
(message handling does nothing). Methods are provided for reading and
updating the interface table and the route table, as well as for unicast
and multicast routing.

\subsection*{Parameters}

The \nedtype{IPv4RoutingTable} module has the following parameters:

\begin{itemize}
  \item \fpar{routerId}: for routers, the router id using IPv4 address dotted notation;
        specify ``auto'' to select the highest interface address; should be left empty ``''
        for hosts
  \item \fpar{IPForward}: turns IP forwarding on/off (It is always \fkeyword{true}
                          in a \nedtype{Router} and is \fkeyword{false} by default
                          in a \nedtype{StandardHost}.)
  \item \fpar{forwardMulticast}: turns multicast IP forwarding on/off. Default is \fkeyword{false},
  should be set to \fkeyword{true} in multicast routers.
  \item \fpar{routingFile}: name of routing file that configures IP addresses and routes of the node
  containing this routing table. Its format is described in section \ref{subsec:routing_files}.
\end{itemize}

\begin{warning}
The \fpar{routingFile} parameter is obsolete. The prefered method for network configuration
is to use \nedtype{IPv4NetworkConfigurator}. The old config files should be replaced
with the XML configuration of \nedtype{IPv4NetworkConfigurator}. Section \ref{subsec:ipv4configurator}
describes the format of the new configuration files.
\end{warning}

% FIXME (#467) IPv4RoutingTable::invalidateCache() should clear localBroadcastAddresses.
% IPv4RoutingTable::findBestMatchingRoute() should search in this order:
%          1. host routes (exact match)
%          2. network routes (longest match)
%          3. default routes (round robin)
%   It is ok, if host routes has 255.255.255.255 netmask, and default has 0.0.0.0 netmask.
% FIXME IPv4RoutingTable::findBestMatchingRoute() if(...MANET...) branch always set bestRoute to NULL,
%       because if there were exact match, it would have been choosen in the previous loop.

\section{The ICMP module}

The Internet Control Message Protocol (ICMP) is the error reporting and
diagnostic mechanism of the Internet.
It uses the services of IP, so it is a transport layer protocol, but unlike
TCP or UDP it is not used to transfer user data. It can not be separated
from the IP, because the routing errors are reported by ICMP.

The \nedtype{ICMP} module can be used to send error messages and ping
request. It can also respond to incoming ICMP messages.

Each ICMP message is encapsulated within an IP datagram, so its delivery
is unreliable.

\begin{center}
\begin{bytefield}{32}
\bitheader{0,7,8,15,31} \\
\bitbox{8}{Type} &
\bitbox{8}{Code} &
\bitbox{16}{Checksum} \\
\bitbox{32}{Rest of header} \\
\wordbox{2}{Internet Header + 8 bytes of Original Datagram}
\end{bytefield}
\end{center}

The corresponding message class (\msgtype{ICMPMessage}) contains only
the Type and Code fields. The message encapsulates the IP packet that
triggered the error, or the data of the ping request/reply.

% FIXME type=PARAMETER_PROBLEM, code=0: missing Pointer field from ICMPMessage
%            REDIRECT: Gateway Internet Address
%            ECHO_REQUEST, ECHO_REPLY: Identifier, Sequence Number
%            TIMESTAMP_REQUEST, TIMESTAMP_REPLY: Identifier, Sequence Number, Originate Timestamp, Receive Timestamp, Transmit Timestamp

% FIXME wrong type codes for ICMP_DESTINATION_UNREACHABLE (3), ICMP_ECHO_REQUEST (8), ICMP_ECHO_REPLY (0), ICMP_TIMESTAMP_REQUEST (13), ICMP_TIMESTAMP_REPLY (14)

% FIXME ICMP header serializer handles only ICMP_ECHO_REQUEST, ICMP_ECHO_REPLY, ICMP_DESTINATION_UNREACHABLE, ICMP_TIME_EXCEEDED
%       ICMP header deserializer handles only ICMP_ECHO_REQUEST, ICMP_ECHO_REPLY

% FIXME ICMP error should not be send if the original datagram
%         1. is an ICMP error
%         2. was sent to a broadcast or multicast address
%         3. datagram was sent with a link-layer broadcast
%         4. a fragment other than the first
%         5. a datagram whose source address is 0.0.0.0, 127.*.*.*, broadcast or multicast address
%      currently only the 1. and half of 2. checked


% \section{The IGMP module}

\section{The ARP module}

The \nedtype{ARP} module implements the Address Resolution Protocol (RFC826).
The ARP protocol is designed to translate a local protocol address
to a hardware address. Altough the ARP protocol can be used with
several network protocol and hardware addressing schemes, in practice
they are almost always IPv4 and 802.3 addresses. The INET implementation
of the ARP protocol (the \nedtype{ARP} module) supports only
IP address $\rightarrow$ MAC address translation.

If a node wants to send an IP packet to a node whose MAC address is unknown,
it broadcasts an ARP frame on the Ethernet network.
In the request its publish its own IP and
MAC addresses, so each node in the local subnet can update their mapping.
The node whose MAC address was requested will respond with an ARP frame
containing its own MAC address directly to the node that sent the
request. When the original node receives the ARP response, it updates
its ARP cache and sends the delayed IP packet using the learned MAC address.

The frame format of the ARP request and reponse is shown in Figure \ref{fig:ARP_frame}.
In our case the HTYPE (hardware type), PTYPE (protocol type), HLEN (hardware address length)
and PLEN (protocol address length) are constants: HTYPE=Ethernet (1), PTYPE=IPv4 (2048), HLEN=6,
PLEN=4. The OPER (operation) field is 1 for an ARP request and 2 for an ARP response.
The SHA field contains the 48-bit hardware address of the sender, SPA field is
the 32-bit IP address of the sender; THA and TPA are the addresses of the target.
The message class corresponding to the ARP frame is \msgtype{ARPPacket}.
In this class only the OPER, SHA, SPA, THA and TPA fields are stored.
The length of an \msgtype{ARPPacket} is 28 bytes.

\begin{figure}[h]
\begin{center}
\label{fig:ARP_frame}
\begin{bytefield}{16}
\bitheader{0,7,8,15} \\
\bitbox{16}{HTYPE} \\
\bitbox{16}{PTYPE} \\
\bitbox{8}{HLEN} &
\bitbox{8}{PLEN} \\
\bitbox{16}{OPER} \\
\wordbox{3}{SHA} \\
\wordbox{2}{SPA} \\
\wordbox{3}{THA} \\
\wordbox{2}{TPA} \\
\end{bytefield}
\caption{ARP frame}
\end{center}
\end{figure}

The \nedtype{ARP} module receives IP datagrams and ARP responses from \nedtype{IPv4}
on the \ttt{ipIn} gate and transmits IP datagrams and ARP requests on the \ttt{nicOut[]} gates
towards the network interface cards. ARP broadcasts the requests on the local network,
so the NIC's entry in the \nedtype{InterfaceTable} should have \ffunc{isBroadcast()} flag
set in order to participate in the address resolution.

The incoming IP packet should have an attached \cppclass{IPv4RoutingDecision} control
info containing the IP address of the next hop. The next hop can be either an
IPv4 broadcast/multicast or a unicast address. The corresponding MAC addresses
can be computed for broadcast and multicast addresses (RFC 1122, 6.4); unicast
addresses are mapped using the ARP procotol.

If the hardware address is found
in the ARP cache, then the packet is transmitted to the addressed interface immediately.
Otherwise the packet is queued and an address resolution takes place.
The \nedtype{ARP} module creates an \msgtype{ARPPacket} object, sets the sender
MAC and IP address to its own address, sets the destination IP address
to the address of the target of the IP datagram, leave the destination MAC address
blank and broadcasts the packet on each network interface with broadcast capability.
Before sending the ARP packet, it retransmission a timer. If the timer expires,
it will retransmit the ARP request, until the maximum retry count is reached.
If there is no response to the ARP request, then the address resolution fails,
and the IP packet is dropped from the queue. Otherwise the MAC address of the
destination is learned and the IP packet can be transmitted on the corresponding
interface.

When an ARP packet is received on the \ttt{ipIn} gate, and the sender's IP
is already in the ARP cache, it is updated with the information in the ARP frame.
Then it is checked that the destination IP of the packet matches with our
address. In this case a new entry is created with the sender addresses in the
ARP cache, and if the packet is a request a response is created and sent directly
to the originator. If proxy ARP is enabled, the request can be responded
with our MAC address if we can route IP packets to the destination.

Usually each \nedtype{ARP} module maintains a local ARP cache.
However it is possible to use a global cache. The global cache is filled
in with entries of the IP and MAC addresses of the known interfaces
when the ARP modules are initiated (at simulation time 0).
\nedtype{ARP} modules that are using the global ARP cache
never initiate an address resolution; if an IP address not
found in the global cache, the simulation stops with an error.
However they will respond to ARP request, so the simulation can
be configured so that some \nedtype{ARP}s use local, while others
the global cache.

When an entry is inserted or updated in the local ARP cache,
the simulation time saved in the entry. The mapping in the
entry is not used after the configured \fpar{cacheTimeout}
elapsed. This parameter does not affect the entries of
the global cache however.

The module parameters of \nedtype{ARP} are:

\begin{itemize}
  \item \fpar{retryTimeout}: number of seconds ARP waits between retries to resolve an IPv4 address (default is 1s)
  \item \fpar{retryCount}: number of times ARP will attempt to resolve an IPv4 address (default is 3)
  \item \fpar{cacheTimeout}: number of seconds unused entries in the cache will time out (default is 120s)
  \item \fpar{proxyARP}: enables proxy ARP mode (default is \fkeyword{true})
  \item \fpar{globalARP}: use global ARP cache (default is \fkeyword{false})
\end{itemize}

The \nedtype{ARP} module emits four signals:

\begin{itemize}
  \item \ttt{sentReq}: emits 1 each time an ARP request is sent
  \item \ttt{sentReplies}: emits 1 each time an ARP response is sent
  \item \ttt{initiatedResolution}: emits 1 each time an ARP resolution is initiated
  \item \ttt{failedResolution}: emits 1 each time an ARP resolution is failed
\end{itemize}

These signals are recorded as vectors and their counts as scalars.

% TODO watches, animation effects

\section{The IGMP module}

The IGMP module is responsible for distributing the information of
multicast group memberships from hosts to routers. When an interface
of a host joins to a multicast group, it will send an IGMP report
on that interface to routers. It can also send reports when the
interface leaves the multicast group, so it does not want to
receive those multicast datagrams. The IGMP module of multicast
routers processes these IGMP reports: it updates the list of
groups, that has members on the link of the incoming message.

The \nedtype{IIGMP} module interface defines the connections
of IGMP modules.
IGMP reports are transmitted by IP, so the module contains
gates to be connected to the IP module (\ttt{ipIn/ipOut}). The IP
module delivers packets with protocol number 2 to the IGMP module.
However some multicast routing protocols (like DVMRP) also exchange
routing information by sending IGMP messages, so they should be
connected to the \ttt{routerIn/routerOut} gates of the IGMP module.
The IGMP module delivers the IGMP messages not processed by itself
to the connected routing module.

The \nedtype{IGMPv2} module implements version 2 of the IGMP protocol
(RFC 2236). Next we describe its behaviour in host and routers in details.
Note that multicast routers behaves as hosts too, i.e. they are sending
reports to other routers when joining or leaving a multicast group.

\subsection{Host behaviour}

When an interface joins to a multicast group, the host
will send a Membership Report immediately to the group address.
This report is repeated after \fpar{unsolicetedReportInterval} to
cover the possibility of the first report being lost.

When a host's interface leaves a multicast group, and it was
the last host that sent a Membership Report for that group,
it will send a Leave Group message to the all-routers multicast
group (224.0.0.2).

This module also responds to IGMP Queries. When the host
receives a Group-Specific Query on an interface that belongs
to that group, then it will set a timer to a random value
between 0 and Max Response Time of the Query. If the timer
expires before the host observe a Membership Report sent
by other hosts, then the host sends an IGMPv2 Membership Report.
When the host receives a General Query on an interface,
a timer is initialized and a report is sent for each group
membership of the interface.

\subsection{Router behaviour}

Multicast routers maintains a list for each interface containing
the multicast groups that have listeners on that interface.
This list is updated when IGMP Membership Reports and Leave Group
messages arrive, or when a timer expires since the last Query.

When multiple routers are connected to the same link, the one with
the smallest IP address will be the Querier. When other routers
observe that they are Non-Queriers (by receiving an IGMP Query
with a lower source address), they stop sending IGMP Queries
until \fpar{otherQuerierPresentInterval} elapsed since the last
received query.

Routers periodically (\fpar{queryInterval}) send a General Query
on each attached network for which this router is a Querier.
On startup the router sends \fpar{startupQueryCount} queries
separated by \fpar{startupQueryInterval}. A General Query
has unspecified Group Address field, a Max Response Time
field set to \fpar{queryResponseInterval}, and is sent to the
all-systems multicast address (224.0.0.1).

When a router receives a Membership Report, it will add the
reported group to the list of multicast group memberships.
At the same time it will set a timer for the membership
to \fpar{groupMembershipInterval}. Repeated reports restart
the timer. If the timer expires, the router assumes
that the group has no local members, and multicast traffic
is no more forwarded to that interface.

When a Querier receives a Leave Group message for a group,
it sends a Group-Specific Query to the group being left.
It repeats the Query \fpar{lastMemberQueryCount} times in
separated by \fpar{lastMemberQueryInterval} until a Membership
Report is received. If no Report received, then the router
assumes that the group has no local members.

% FIXME IGMPv2 not compatible with IGMPv1 hosts and routers

\subsection{Disabling IGMP}

The IPv4 \nedtype{IPv4NetworkLayer} contains an instance of the IGMP
module. IGMP can be turned off by setting the \fpar{enabled}
parameter to false. When disabled, then no IGMP message
is generated, and incoming IGMP messages are ignored.

\subsection{Parameters}

The following parameters has effects in both hosts and routers:

\begin{itemize}
  \item \fpar{enabled} if \fkeyword{false} then the IGMP module is silent. Default is \fkeyword{true}.
\end{itemize}

These parameters are only used in hosts:

\begin{itemize}
  \item \fpar{unsolicitedReportInterval} the time between repetitions of a
   host's initial report of membership in a group. Default is 10s.
\end{itemize}

Router timeouts are configured by these parameters:

\begin{itemize}
  \item \fpar{robustnessVariable} the IGMP is robust to \fpar{robustnessVariable}-1
   packet losses. Default is 2.
  \item \fpar{queryInterval} the interval between General Queries sent by a Querier.
   Default is 125s.
  \item \fpar{queryResponseInterval} the Max Response Time inserted into General Queries
  \item \fpar{groupMembershipInterval} the amount of time that must pass before
   a multicast router decides there are no more members of a group on a network.
   Fixed to \fpar{robustnessVariable} * \fpar{queryInterval} + \fpar{queryResponseInterval}.
  \item \fpar{otherQuerierPresentInterval} the length of time that must
   pass before a multicast router decides that there is no longer
   another multicast router which should be the querier.
   Fixed to \fpar{robustnessVariable} * \fpar{queryInterval} + \fpar{queryResponseInterval} / 2.
  \item \fpar{startupQueryInterval} the interval between General Queries
   sent by a Querier on startup. Default is \fpar{queryInterval} / 4.
  \item \fpar{startupQueryCount} the number of Queries sent out on startup,
   separated by the \fpar{startupQueryInterval}. Default is \fpar{robustnessVariable}.
  \item \fpar{lastMemberQueryInterval} the Max Response Time inserted into
   Group-Specific Queries sent in response to Leave Group messages, and
   is also the amount of time between Group-Specific Query messages.
   Default is 1s.
  \item \fpar{lastMemberQueryCount} the number of Group-Specific Queries
   sent before the router assumes there are no local members.
   Default is \fpar{robustnessVariable}.
\end{itemize}

\section{The IPv4NetworkLayer module}

The \nedtype{IPv4NetworkLayer} module packs the \nedtype{IP}, \nedtype{ICMP},
\nedtype{ARP}, and \nedtype{IGMP} modules into one compound module.
The compound module defines gates for connecting UDP, TCP, SCTP, RSVP and
OSPF transport protocols. The \ttt{pingIn} and \ttt{pingOut} gates of the
\nedtype{ICMP} module are also available, while its \ttt{errorOut} gate
is connected to an inner \nedtype{ErrorHandling} component that writes
the ICMP errors to the log.

The component can be used in hosts and routers to support IPv4.

\section{The NetworkInfo module}

The \nedtype{NetworkInfo} module can be used to dump detailed information
about the network layer. This module does not send or received messages,
it is invoked by the \nedtype{ScenarioManager} instead. For example
the following \nedtype{ScenarioManager} script dump the routing table
of the \ttt{LSR2} module at simulation time $t=2$ into \ffilename{LSR2\_002.txt}:
\begin{filelisting}
<scenario>
  <at t="2">
    <routing module="NetworkInfo" target="LSR2" file="LSR2_002.txt"/>
  </at>
</scenario>
\end{filelisting}

The module currently support only the \ttt{routing} command which dumps
the routing table. The command has four parameters given as XML attributes:
\begin{itemize}
  \item \ttt{target} the name of the node that owns the routing table to be dumped
  \item \ttt{filename} the name of the file the output is directed to
  \item \ttt{mode} if set to ``a'', the output is appended to the file,
                   otherwise the target is truncated if the file existed
  \item \ttt{compat} if set to ``linux'', then the output is generated
                     in the format of the \ttt{route -n} command of Linux.
                     The output is sorted only if \ttt{compat} is
                     \fkeyword{true}.
\end{itemize}

%%% Local Variables:
%%% mode: latex
%%% TeX-master: "usman"
%%% End:

