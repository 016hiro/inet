\chapter{Using the INET Framework}
\label{cha:usage}

\section{Installation}

There are several ways to install the INET Framework:

\begin{itemize}
  \item Let the OMNeT++ IDE download and install it for you. 
      This is the easiest way. Just accept the offer to install INET
      in the dialog that comes up when you first start the IDE, or
      choose \textit{Help > Install Simulation Models} any time later.
  \item From INET Framework web site, \textit{http://inet.omnetpp.org}. 
      The IDE always installs the last stable version compatible with
      your version of OMNeT++. If you need some other version, they
      are available for download from the web site. Installation
      instructions are also provided there.  
  \item From GitHub. If you have experience with \textit{git}, 
      clone the INET Framework project (\ttt{inet\--frame\-work/inet}), 
      check out the revision of your choice, and follow the INSTALL 
      file in the project root.
\end{itemize}
 
 
\section{Installing INET Extensions}

If you plan to make use of INET extensions (e.g. Veins or SimuLTE),
follow the installation instructions provided with them. 

In the absence of specific instructions, the following procedure usually works: 

\begin{itemize}
 \item First, check if the project root contains a file named \ttt{.project}.
 \item If it does, then the project can be imported into the IDE (use \textit{File > Import >
    General > Existing Project} into workspace). make sure that the project is recognized
    as an OMNeT++ project (the \textit{Project Properties} dialog contains a page
    titled \textit{OMNeT++}), and it lists the INET project as dependency 
    (check the \textit{Project References} page in the \textit{Project Properties} dialog).
 \item If there is no \ttt{.project} file, you can create an empty OMNeT++
    project using the \textit{New OMNeT++ Project} wizard in \textit{File > New}, 
    add the INET project as dependency using the \textit{Project References} page 
    in the \textit{Project Properties} dialog, and copy the source files into the project.
\end{itemize}

\section{INET as an OMNeT++-based simulation framework}

The INET Framework builds upon OMNeT++, and uses the same concept: modules
that communicate by message passing. Hosts, routers, switches and other
network devices are represented by OMNeT++ compound modules. These compound
modules are assembled from simple modules that represent protocols,
applications, and other functional units. A network is again an OMNeT++
compound module that contains host, router and other modules. The external
interfaces of modules are described in NED files. NED files describe the
parameters and gates (i.e. ports or connectors) of modules, and also the
submodules and connections (i.e. netlist) of compound modules.

Modules are organized into hierarchical \textit{packages} that directly map to
folders under \ttt{src/}. Packages in
INET are organized roughly according to OSI layers; the top packages
include \ttt{inet.applications}, \ttt{inet.transportlayer},
\ttt{inet.networklayer}, \ttt{inet.linklayer}, and \ttt{inet.physicallayer}. 
Other packages are \ttt{inet.routing}, \ttt{inet.mobility}, \ttt{inet.power}, 
\ttt{inet.environment}, and \ttt{inet.node}. These packages correspond to the
\ttt{src/applications/}, \ttt{src/transportlayer/}, etc. directories in the
INET source tree. (The \ttt{src/inet/} directory corresponds to the \ttt{inet} 
package, as defined by the \ttt{src/inet/package.ned} file.) Subdirectories
within the top packages usually correspond to concrete protocols or protocol
families. The implementations of simple modules are C++ classes with the same
name, with the source files placed in the same directory as the NED file.

Protocol headers and packet formats are described in message definition
files (msg files), which are translated into C++ classes by OMNeT++'s
\textit{opp\_msgc} tool. The generated message classes subclass from OMNeT++'s
\ttt{cPacket} or \ttt{cMessage} classes.

\section{Setting up simulations}

TODO Examples and practical  guidance: see examples, showcases, tictoc and INET tutorials.


%%% Local Variables:
%%% mode: latex
%%% TeX-master: "usman"
%%% End:

