\chapter{Networks}
\label{cha:networks}

\section{Overview}

%TODO: wired, wireless, mixed wired/wireless, various topologies + generated, hierarchical, parametric
%TODO: ethernet networks, mpls networks, vpn, tunneling, PPP networks, sensor networks

INET heavily builds upon the modular architecture of OMNeT++. It provides 
numerous domain specific and highly parameterizable components which can be
combined in many ways. The primary means of building large custom network 
simulations in INET is the composition of existing models with custom models,
starting from small components and gradually forming ever larger ones up until
the composition of the network. Users are not required to have programming 
experience to create simulations unless they also want to implement 
their own protocols, for example.

As INET is an OMNeT++-based framework, users mainly use NED to describe the
model topology, and ini files to provide configuration.\footnote{Some 
components require additional configuration to be provided as separate
files, e.g. in XML.} 

\section{Assembling Simulations}

Assembling an INET simulation starts with defining a module representing
the network. Networks are compound modules which contain network nodes,
automatic network configurators, and sometimes additionally transmission
medium, physical environment, various visualizer, and other infrastructure
related modules. Networks also contain connections between network nodes
representing cables. Large hierarchical networks may be further organized
into compound modules to directly express the hierarchy.

Network nodes communicate at the network level by exchanging OMNeT++ messages 
which are the abstract representations of physical signals on the 
transmission medium.  Signals are either sent through OMNeT++ connections 
in the wired case, or sent directly to the gate of the receiving network node 
in the wireless case. Signals encapsulate INET-specific packets that represent 
the transmitted digital data. Packets are further divided into chunks that
provide alternative representations for smaller pieces of data (e.g. 
protocol headers, application data).

\section{Wired Networks}

Wired network connections, for example \protocol{Ethernet} cables, are
represented with standard OMNeT++ connections using the
\nedtype{DatarateChannel} NED type. The channel's \nedtype{datarate} and
\nedtype{delay} parameters must be provided for all wired connections.

The following example shows how straightforward it is to create a model for
a simple wired network. This network contains a server connected to a router
using \protocol{PPP}, which in turn is connected to a switch using
\protocol{Ethernet}. The network also contains a parameterizable number of
clients, all connected to the switch forming a star topology. The utilized
network nodes are all predefined modules in INET. To avoid the manual
configuration of IP addresses and routing tables, an automatic network
configurator is also included.

\nedsnippet{WiredNetworkExample}{Wired network example}

In order to run a simulation using the above network, an OMNeT++ INI file must
be created. The INI file selects the network, sets its number of clients
parameter, and configures a simple \protocol{TCP} application for each
client. The server is configured to have a \protocol{TCP} application which
echos back all data received from the clients individually.

\inisnippet{WiredNetworkConfigurationExample}{Wired network configuration example}

When the above simulation is run, each client application connects to the
server using a \protocol{TCP} socket. Then each one of them sends 1MB of
data, which in turn is echoed back by the server, and the simulation
concludes. The default statistics are written to the \texttt{results}
folder of the simulation for later analysis.

\section{Wireless Networks}

%TODO: AccessPoint, WirelessHost infrastructure mode

Wireless network connections are not modeled with OMNeT++ connections due the
dynamically changing nature of connectivity. For wireless networks, an
additional module representing the transmission medium, is required to
maintain connectivity information.

Building a simple wireless network is somewhat different, because wireless
nodes don't use OMNeT++ connections. In this case, the network contains an
additional module which represents the transmission medium.

\nedsnippet{WirelessNetworkExample}{Wireless network example}

For the above network, the INI file configures node mobility using a
stochastic model. In wireless simulations, some form of a mobility model is
essential to provide positions for the transmission medium during the
computation of signal propagation and path loss. In addition, each ad hoc
node is configured to include a simple ping application.

\inisnippet{WirelessNetworkConfigurationExample}{Wireless network configuration example}

When the above simulation is run, each ad hoc node periodically sends an
ICMP echo requests to the first node. This simulation runs indefinitely
while it continuously prints the usual ping round trip time report on the
standard output.

\section{Ethernet Networks}

%TODO: EtherHub, EtherBus, EtherSwitch, EtherHost

\section{MPLS Networks}

%TODO: LdpRouter, RsvpRouter

\section{Mobile Ad hoc Networks}

%TODO: AdhocHost, ManetRouter, AodvRouter, DymoRouter, GpsrRouter

\section{Sensor Networks}

%TOOD: SensorNode, SensorGateway

\section{Virtual Private Networks}

%TOOD: VpnIngressNode, VpnEgressNode

\section{Mixed Networks}

%TODO: wired + wireless

\section{Complex Topologies}

%TODO: wizards, generated topologies

\section{Hierarchical Networks}

%TODO: nested compound modules

\section{//////////////////////////////////////////////////////////////////////}


\section{Connecting Protocols}

TODO describe how INET protocols are connected to each other using the built-in message dispatching mechanism.

\section{Applications}
\label{subsec:applications}

TODO describe how INET models running applications as modules: ping, connectionless traffic, connection oriented traffic, voip, video, etc.

\section{Protocols} % TODO: inline subsubsections one level up
\label{subsec:protocols}

TODO describe how INET models data and control plane protocols as modules in general

\section{Routing Protocols}

TODO briefly describe how INET models routing protocols: BGP, OSPF, RIP, AODV, etc.

%%% Local Variables:
%%% mode: latex
%%% TeX-master: "usman"
%%% End:


