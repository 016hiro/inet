\chapter{Networks}
\label{cha:networks}

%
% This chapter provides practical guidance on how to put together various 
% networks from the built-in node models and how to configure them,
% WITHOUT LOOKING AT THE INTERNALS OF THOSE NODES.
%

\section{Overview}

%TODO: wired, wireless, mixed wired/wireless, various topologies + generated, hierarchical, parametric
%TODO: ethernet networks, mpls networks, vpn, tunneling, PPP networks, sensor networks

INET heavily builds upon the modular architecture of OMNeT++. It provides 
numerous domain specific and highly parameterizable components which can be
combined in many ways. The primary means of building large custom network 
simulations in INET is the composition of existing models with custom models,
starting from small components and gradually forming ever larger ones up until
the composition of the network. Users are not required to have programming 
experience to create simulations unless they also want to implement 
their own protocols, for example.

Assembling an INET simulation starts with defining a module representing
the network. Networks are compound modules which contain network nodes,
automatic network configurators, and sometimes additionally transmission
medium, physical environment, various visualizer, and other infrastructure
related modules. Networks also contain connections between network nodes
representing cables. Large hierarchical networks may be further organized
into compound modules to directly express the hierarchy.

There are no predefined networks in INET, because it is very easy to create
one, and because of the vast possibilities. However, the OMNeT++ IDE provides
several topology generator wizards for advanced scenarios.

As INET is an OMNeT++-based framework, users mainly use NED to describe the
model topology, and ini files to provide configuration.\footnote{Some 
components require additional configuration to be provided as separate
files, e.g. in XML.} 

\section{Built-in Network Nodes and Other Top-Level Modules}

INET provides several pre-assembled network nodes with carefully selected
components. They support customization via parameters and parametric
submodule types, but they are not meant to be universal. Sometimes it may
be necessary to create special network node models for particular
simulation scenarios. In any case, the following list gives a taste of the
built-in network nodes.

\begin{itemize}
  \item \nedtype{StandardHost} contains the most common Internet protocols:
     \protocol{UCP}, \protocol{TDP}, \protocol{IPv4}, \protocol{IPv6},
     \protocol{Ethernet}, \protocol{IEEE 802.11}. It also supports an
     optional mobility model, optional energy models, and any number of
     applications which are entirely configurable from INI files.
  \item \nedtype{EtherSwitch} models an \protocol{Ethernet} switch containing
     a relay unit and one MAC unit per port.
  \item \nedtype{Router} provides the most common routing protocols:
     \protocol{OSPF}, \protocol{BGP}, \protocol{RIP}, \protocol{PIM}.
  \item \nedtype{AccessPoint} models a Wifi access point with multiple
     \protocol{IEEE 802.11} network interfaces and multiple \protocol{Ethernet}
     ports.
  \item \nedtype{WirelessHost} provides a network node with one (default)
     \protocol{IEEE 802.11} network interface in infrastructure mode,
     suitable for using with an \nedtype{AccessPoint}.
  \item \nedtype{AdhocHost} is a \nedtype{WirelessHost} with the network
     interface configured in ad-hoc mode and forwarding enabled.
  \item \nedtype{AodvRouter} is similar to an \nedtype{AdhocHost} with
     an additional \protocol{AODV} protocol.
\end{itemize}

Network nodes communicate at the network level by exchanging OMNeT++ messages 
which are the abstract representations of physical signals on the 
transmission medium.  Signals are either sent through OMNeT++ connections 
in the wired case, or sent directly to the gate of the receiving network node 
in the wireless case. Signals encapsulate INET-specific packets that represent 
the transmitted digital data. Packets are further divided into chunks that
provide alternative representations for smaller pieces of data (e.g. 
protocol headers, application data).

Additionally, there are components that occur on network level, but they
are not models of physical network nodes. They are necessary 
to model other aspects. Some of them are:

\begin{itemize}
  \item \nedtype{ScenarioManager} allows scripted scenarios, such
     as timed failure and recovery of network nodes.
  \item \nedtype{Ipv4NetworkConfigurator} assigns IP addresses 
     to hosts and routers, and sets up static routing.
\item \nedtype{XXXVisualizer} is ...
\item \nedtype{Medium} is ...
\item ...
\item \nedtype{PhysicalEnvironment}
\item \nedtype{L2NetworkConfigurator}
\item \nedtype{MoBanCoordinator}
\item \nedtype{GenericNetworkConfigurator}
\item \nedtype{ApskLayeredScalarRadioMedium}
\item \nedtype{IRadioMedium}
\item \nedtype{Ieee80211LayeredScalarRadioMedium}
\item \nedtype{Ieee80211DimensionalRadioMedium}
\item \nedtype{Ieee80211ScalarRadioMedium}
\item \nedtype{UnitDiskRadioMedium}
\item \nedtype{IntegratedCanvasVisualizer}

\end{itemize}

TODO revise and extend list

\section{Network Examples}

\subsection{Wired Networks}

Wired network connections, for example \protocol{Ethernet} cables, are
represented with standard OMNeT++ connections using the
\nedtype{DatarateChannel} NED type. The channel's \nedtype{datarate} and
\nedtype{delay} parameters must be provided for all wired connections.

The following example shows how straightforward it is to create a model for
a simple wired network. This network contains a server connected to a router
using \protocol{PPP}, which in turn is connected to a switch using
\protocol{Ethernet}. The network also contains a parameterizable number of
clients, all connected to the switch forming a star topology. The utilized
network nodes are all predefined modules in INET. To avoid the manual
configuration of IP addresses and routing tables, an automatic network
configurator is also included.

\nedsnippet{WiredNetworkExample}{Wired network example}

In order to run a simulation using the above network, an OMNeT++ INI file must
be created. The INI file selects the network, sets its number of clients
parameter, and configures a simple \protocol{TCP} application for each
client. The server is configured to have a \protocol{TCP} application which
echos back all data received from the clients individually.

\inisnippet{WiredNetworkConfigurationExample}{Wired network configuration example}

When the above simulation is run, each client application connects to the
server using a \protocol{TCP} socket. Then each one of them sends 1MB of
data, which in turn is echoed back by the server, and the simulation
concludes. The default statistics are written to the \texttt{results}
folder of the simulation for later analysis.

\subsection{Wireless Networks}

%TODO: AccessPoint, WirelessHost infrastructure mode

Wireless network connections are not modeled with OMNeT++ connections due the
dynamically changing nature of connectivity. For wireless networks, an
additional module representing the transmission medium, is required to
maintain connectivity information.

Building a simple wireless network is somewhat different, because wireless
nodes don't use OMNeT++ connections. In this case, the network contains an
additional module which represents the transmission medium.

\nedsnippet{WirelessNetworkExample}{Wireless network example}

For the above network, the INI file configures node mobility using a
stochastic model. In wireless simulations, some form of a mobility model is
essential to provide positions for the transmission medium during the
computation of signal propagation and path loss. In addition, each ad hoc
node is configured to include a simple ping application.

\inisnippet{WirelessNetworkConfigurationExample}{Wireless network configuration example}

When the above simulation is run, each ad hoc node periodically sends an
ICMP echo requests to the first node. This simulation runs indefinitely
while it continuously prints the usual ping round trip time report on the
standard output.

\subsection{Ethernet Networks}

%TODO: EtherHub, EtherBus, EtherSwitch, EtherHost

\subsection{MPLS Networks}

%TODO: LdpRouter, RsvpRouter

\subsection{Mobile Ad hoc Networks}

%TODO: AdhocHost, ManetRouter, AodvRouter, DymoRouter, GpsrRouter

\subsection{Sensor Networks}

%TOOD: SensorNode, SensorGateway

\subsection{Virtual Private Networks}

%TOOD: VpnIngressNode, VpnEgressNode

\subsection{Mixed Networks}

%TODO: wired + wireless


\section{Frequent Tasks (How To...)}

Quick and somewhat superficial advice to many practical tasks.

\subsection{Automatic Wired Interfaces}

In many wired network simulations, the number of wired interfaces need not
be manually configured, because it can be automatically inferred from the
actual number of connections between network nodes.

\nedsnippet{AutomaticWiredInterfacesExample}{Automatic wired interfaces
example}

\subsection{Multiple Wireless Interfaces}

All built-in wireless network nodes support multiple wireless interfaces,
but only one is enabled by default.

\inisnippet{MultipleWirelessInterfacesExample}{Multiple wireless interfaces
example}

\subsection{Traffic Generation}

TODO scripted, synthetic, CBR, VBR, trace-based, .... 
TODO app[]; other ways 

\subsection{Specifying Addresses}

Nearly all application layer modules, but several other components as well,
have parameters that specify network addresses. They typically accept
addresses given with any of the following syntax variations:

\begin{itemize}
  \item literal IPv4 address: \ttt{"186.54.66.2"}
  \item literal IPv6 address: \ttt{"3011:7cd6:750b:5fd6:aba3:c231:e9f9:6a43"}
  \item module name: \ttt{"server"}, \ttt{"subnet.server[3]"}
  \item interface of a host or router: \ttt{"server/eth0"}, \ttt{"subnet.server[3]/eth0"}
  \item IPv4 or IPv6 address of a host or router: \ttt{"server(ipv4)"},
      \ttt{"subnet.server[3](ipv6)"}
  \item IPv4 or IPv6 address of an interface of a host or router:
      \ttt{"server/eth0(ipv4)"}, \ttt{"subnet.server[3]/eth0(ipv6)"}
\end{itemize}

TODO ini example


\subsection{Node Failure and Recovery}

\subsection{Enabling Dual IP Stack}

All built-in network nodes support dual Internet protocol stacks, that is
both \protocol{Ipv4} and \protocol{Ipv6} are available. They are also
supported by transport layer protocols, link layer protocols, and most
applications. Only \protocol{Ipv4} is enabled by default, so in order to
use \protocol{Ipv6}, it must be enabled first, and an application
supporting \protocol{Ipv6} (e.g., \nedtype{PingApp} must be used). The
following example shows how to configure two ping applications in a single
node where one is using an \protocol{Ipv4} and the other is using an
\protocol{Ipv6} destination address.

\inisnippet{DualStackExample}{Dual stack example}

\subsection{Enabling Packet Forwarding}

In general, network nodes don't forward packets by default, only
\nedtype{Router} and the like do. Nevertheless, it's possible to enable
packet forwarding as simply as flipping a switch.

\inisnippet{ForwardingExample}{Forwarding example}

\subsection{Adding Routing Protocols}

TODO internet routing and ad hoc routing

\subsection{Node Mobility}

TODO

\subsection{Topology Generation}

TODO: wizards, generated topologies

\subsection{Hierarchical Networks}

TODO: nested compound modules

%%% Local Variables:
%%% mode: latex
%%% TeX-master: "usman"
%%% End:


