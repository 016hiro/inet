\chapter{The UDP Model}
\label{cha:udp}


\section{Overview}

The UDP protocol is a very simple datagram transport protocol, which
basically makes the services of the network layer available to the applications.
It performs packet multiplexing and demultiplexing to ports and some basic
error detection only.

The frame format as described in RFC768:

\begin{center}
\begin{bytefield}{32}
\bitheader{0,7,8,15,16,23,24,31} \\
\bitbox{16}{Source Port} &
\bitbox{16}{Destination Port} \\
\bitbox{16}{Length} &
\bitbox{16}{Checksum} \\
\wordbox{3}{Data}
\end{bytefield}
\end{center}

The ports represents the communication end points that are allocated by the
applications that want to send or receive the datagrams. The ``Data'' field
is the encapsulated application data, the ``Length'' and ``Checksum'' fields
are computed from the data.

The INET framework contains an \nedtype{Udp} module that performs the encapsulation/decapsulation
of user packets, an \nedtype{UdpSocket} class that provides the application the usual
socket interface, and several sample applications.

These components implement the following statndards:
\begin{itemize}
\item RFC768: User Datagram Protocol
\item RFC1122: Requirements for Internet Hosts -- Communication Layers
\end{itemize}

\section{The UDP module}

The UDP protocol is implemented by the \nedtype{Udp} simple module.
There is a module interface (\nedtype{IUdp}) that defines the gates of the
\nedtype{Udp} component. In the \nedtype{StandardHost} node, the UDP component
can be any module implementing that interface.

Each UDP module has gates to connect to the IPv4 and IPv6 network layer
(ipIn/ipOut and ipv6In/ipv6Out), and a gate array to connect to the applications
(appIn/appOut).

The UDP module can be connected to several applications, and each application
can use several sockets to send and receive UDP datagrams.

%%% Local Variables:
%%% mode: latex
%%% TeX-master: "usman"
%%% End:

