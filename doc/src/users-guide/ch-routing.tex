\chapter{Internet Routing}
\label{cha:routing}

\section{Overview}
\label{sec:routing:overview}

INET Framework has models for several internet routing protocols, including
RIP, OSPF and BGP.

The easiest way to add routing to a network is to use the \nedtype{Router}
NED type for routers. \nedtype{Router} contains a conditional instance
for each of the above protocols. These submodules can be enabled by
setting the \ttt{hasRIP}, \ttt{hasOSPF} and/or \ttt{hasBGP} parameters to
\ttt{true}.

Example:

\begin{inifile}
**.hasRIP = true
\end{inifile}

There are also NED types called \nedtype{RipRouter}, \nedtype{OspfRouter},
\nedtype{BgpRouter}, which are all \nedtype{Router}s with appropriate
routing protocol enabled.

\section{RIP}
\label{sec:routing:rip}

RIP (Routing Information Protocol) is a distance-vector routing protocol
which employs the hop count as a routing metric. RIP prevents routing loops
by implementing a limit on the number of hops allowed in a path from source
to destination.

The \nedtype{Rip} module implements distance vector routing as
specified in RFC 2453 (RIPv2) and RFC 2080 (RIPng). Configuration
can be specified in an XML file that can be specified in the
\ttt{ripConfig} parameter.

The \nedtype{Rip} module has the following parameters:

\begin{itemize}
  \item \ttt{mode}: either "RIPv2" (RFC 2453) or "RIPng" (RFC 2080)
  \item \ttt{routingTableModule}: path to the routing table module
        e.g. \ttt{'\^{}.ipv4.routingTable'}
  \item \ttt{ripConfig}: an XML configuration file containing per-interface parameters
\end{itemize}

The following example configures a \nedtype{Router} module to use RIPv2:

\begin{inifile}
**.hasRIP = true
**.mode = "RIPv2"
**.ripConfig = xmldoc("RIPConfig.xml")
\end{inifile}

The configuration file specifies the per interface parameters.
Each \ttt{<interface>} element configures one or more interfaces;
the \ttt{hosts}, \ttt{names}, \ttt{towards}, \ttt{among} attributes
select the configured interfaces (in a similar way as with
\nedtype{Ipv4NetworkConfigurator} \ref{cha:network-autoconfiguration}).

Additional attributes:

\begin{itemize}
  \item \ttt{metric}: metric assigned to the link, default value is 1.
        This value is added to the metric of a learned route,
        received on this interface. It must be an integer in
        the [1,15] interval.
  \item \ttt{mode}: mode of the interface.
\end{itemize}

The mode attribute can be one of the following:

\begin{itemize}
  \item \ttt{'NoRIP'}: no RIP messages are sent or received on this interface.
  \item \ttt{'NoSplitHorizon'}: no split horizon filtering; send all routes to
        neighbors.
  \item \ttt{'SplitHorizon'}: do not sent routes whose next hop is the neighbor.
  \item \ttt{'SplitHorizonPoisenedReverse'} (default): if the next hop is the neighbor, then
  set the metric of the route to infinity.
\end{itemize}

The following example sets the link metric between router
\ttt{R1} and \ttt{RB} to 2, while all other links will have a metric of 1.

\begin{XML}
<RIPConfig>
  <interface among="R1 RB" metric="2"/>
  <interface among="R? R?" metric="1"/>
</RIPConfig>
\end{XML}

\section{OSPF}
\label{sec:routing:ospf}

OSPF (Open Shortest Path First) is a routing protocol for IP networks.
It uses a link state routing (LSR) algorithm and falls into the group
of interior gateway protocols (IGPs), operating within a single
autonomous system (AS).

\nedtype{OspfRouter} is a \nedtype{Router} with the OSPF protocol enabled.

The \nedtype{Ospf} module implements OSPF protocol version 2. Areas and routers
can be configured using an XML file specified by the \ttt{ospfConfig} parameter.
Various parameters for the network interfaces can be specified also in the XML
file or as a parameter of the \nedtype{Ospf} module.

\begin{inifile}
**.ospf.ospfConfig = xmldoc("ASConfig.xml")
**.ospf.helloInterval = 12s
**.ospf.retransmissionInterval = 6s
\end{inifile}

The \ttt{<OSPFASConfig>} root element may contain \ttt{<Area>} and \ttt{<Router>}
elements with various attributes specifying the parameters for the network
interfaces. Most importantly \ttt{<Area>} contains \ttt{<AddressRange>} elements
enumerating the network addresses that should be advertized by the protocol.
Also \ttt{<Router>} elements may contain data for configuring various pont-to-point
or broadcast interfaces.

\begin{XML}
<?xml version="1.0"?>
<OSPFASConfig xmlns:xsi="http://www.w3.org/2001/XMLSchema-instance" xsi:schemaLocation="OSPF.xsd">
  <!-- Areas -->
  <Area id="0.0.0.0">
    <AddressRange address="H1" mask="H1" status="Advertise" />
    <AddressRange address="H2" mask="H2" status="Advertise" />
    <AddressRange address="R1>R2" mask="R1>R2" status="Advertise" />
    <AddressRange address="R2>R1" mask="R2>R1" status="Advertise" />
  </Area>
  <!-- Routers -->
  <Router name="R1" RFC1583Compatible="true">
    <BroadcastInterface ifName="eth0" areaID="0.0.0.0" interfaceOutputCost="1" routerPriority="1" />
    <PointToPointInterface ifName="eth1" areaID="0.0.0.0" interfaceOutputCost="2" />
  </Router>
  <Router name="R2" RFC1583Compatible="true">
    <PointToPointInterface ifName="eth0" areaID="0.0.0.0" interfaceOutputCost="2" />
    <BroadcastInterface ifName="eth1" areaID="0.0.0.0" interfaceOutputCost="1" routerPriority="2" />
  </Router>
</OSPFASConfig>
\end{XML}

\section{BGP}
\label{sec:routing:bgp}

BGP (Border Gateway Protocol) is a standardized exterior gateway protocol
designed to exchange routing and reachability information among
autonomous systems (AS) on the Internet.

\nedtype{BgpRouter} is a \nedtype{Router} with the BGP protocol enabled.

The \nedtype{Bgp} module implements BGP Version 4. The model implements
RFC 4271, with some limitations. Autonomous Systems and rules can be
configured in an XML file that can be specified in the \ttt{bgpConfig}
parameter.

\begin{inifile}
**.bgpConfig = xmldoc("BGPConfig.xml")
\end{inifile}

The configuration file may contain \ttt{<TimerParams>}, \ttt{<AS>} and
\ttt{Session} elements at the top level.

\begin{itemize}
  \item \ttt{<TimerParams>}: allows specifying various timing parameters
  for the routers.
  \item \ttt{<AS>}: defines Autonomous Systems, routers and rules to be applied.
  \item \ttt{<Session>}: specifies open sessions between edge routers. It must
  contain exactly two \ttt{<Router exterAddr="x.x.x.x"/>} elements.
\end{itemize}

\begin{XML}
<BGPConfig xmlns:xsi="http://www.w3.org/2001/XMLSchema-instance"
  xsi:schemaLocation="BGP.xsd">

  <TimerParams>
    <connectRetryTime> 120 </connectRetryTime>
    <holdTime> 180 </holdTime>
    <keepAliveTime> 60 </keepAliveTime>
    <startDelay> 15 </startDelay>
  </TimerParams>

  <AS id="60111">
    <Router interAddr="172.1.10.255"/> <!--Router A1-->
    <Router interAddr="172.1.20.255"/> <!--Router A2-->
  </AS>

  <AS id="60222">
    <Router interAddr="172.10.4.255"/> <!--Router B-->
  </AS>

  <AS id="60333">
    <Router interAddr="172.13.1.255"/> <!--Router C1-->
    <Router interAddr="172.13.2.255"/> <!--Router C2-->
    <Router interAddr="172.13.3.255"/> <!--Router C3-->
    <Router interAddr="172.13.4.255"/> <!--Router C4-->
    <DenyRouteOUT Address="172.10.8.0" Netmask="255.255.255.0"/>
    <DenyASOUT> 60111 </DenyASOUT>
  </AS>

  <Session id="1">
    <Router exterAddr="10.10.10.1" > </Router> <!--Router A1-->
    <Router exterAddr="10.10.10.2" > </Router> <!--Router C1-->
  </Session>

  <Session id="2">
    <Router exterAddr="10.10.20.1" > </Router> <!--Router A2-->
    <Router exterAddr="10.10.20.2" > </Router> <!--Router B-->
  </Session>

  <Session id="3">
    <Router exterAddr="10.10.30.1" > </Router> <!--Router B-->
    <Router exterAddr="10.10.30.2" > </Router> <!--Router C2-->
  </Session>
</BGPConfig>
\end{XML}

Inside \ttt{<AS>} elements various rules can be sepecified:

\begin{itemize}
  \item DenyRoute: deny route in IN and OUT traffic (\ttt{Address} and
        \ttt{Netmask} attributes must be specified.)
  \item DenyRouteIN : deny route in IN traffic (\ttt{Address} and
        \ttt{Netmask} attributes must be specified.)
  \item DenyRouteOUT: deny route in OUT traffic (\ttt{Address} and
        \ttt{Netmask} attributes must be specified.)
  \item DenyAS: deny routes learned by AS in IN  and OUT traffic (AS id must be
        specified as the body of the element.)
  \item DenyASIN : deny routes learned by AS in IN traffic (AS id must be
        specified as the body of the element.)
  \item DenyASOUT: deny routes learned by AS in OUT traffic (AS id must be
        specified as the body of the element.)
\end{itemize}

%%% Local Variables:
%%% mode: latex
%%% TeX-master: "usman"
%%% End:

