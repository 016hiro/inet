\chapter{The MPLS Models}
\label{cha:mpls}


\section{Overview}

TODO


\section{MPLS/RSVP/LDP Model - Implemented Standards}

The implementation follows those RFCs below:

\begin{itemize}
  \item RFC 2702: Requirements for Traffic Engineering Over MPLS
  \item RFC 2205: Resource ReSerVation Protocol
  \item RFC 3031: Multiprotocol Label Switching Architecture
  \item RFC 3036: LDP Specification
  \item RFC 3209: RSVP-TE Extension to RSVP for LSP tunnels
  \item RFC 2205: RSVP Version 1 - Functional Specification
  \item RFC 2209: RSVP Message processing Version 1
\end{itemize}

\section{The MPLS Module}

TODO

\section{The LDP Module}

TODO

\section{LIB Table File Format}

The format of a LIB table file is:

The beginning of the file should begin with comments. Lines that begin with \# are treated
as comments. An empty line is required after the comments. The "LIB TABLE"
syntax must come next with an empty line. The column headers follow. This header
must be strictly "In-lbl In-intf Out-lbl Out-intf". Column
values are after that with space or tab for field separation.
The following is a sample of lib table file.

\begin{verbatim}
#lib table for MPLS network simulation test
#lib1.table for LSR1 - this is an edge router
#no incoming label for traffic from in-intf 0 &1 - LSR1 is ingress router for those traffic
#no outgoing label for traffic from in_intf 2 &3 - LSR 1 is egress router for those traffic

LIB TABLE:

In-lbl  In-intf         Out-lbl     Out-intf
1       193.233.7.90    1           193.231.7.21
2       193.243.2.1     0           193.243.2.3
\end{verbatim}


\section{The traffic.xml file}

The traffic.xml file is read by the RSVP-TE module (RSVP).
The file must be in the same folder as the executable
network simulation file.

The XML elements used in the "traffic.xml" file:

\begin{itemize}
  \item \ttt{<Traffic></Traffic>} is the root element. It may contain one or more \ttt{<Conn>} elements.
  \item \ttt{<Conn></Conn>} specifies an RSVP session. It may contain the following elements:
  \begin{itemize}
    \item \ttt{<src></src>} specifies sender IP address
    \item \ttt{<dest></dest>} specifies receiver IP address
    \item \ttt{<setupPri></setupPri>} specifies LSP setup priority
    \item \ttt{<holdingPri></holdingPri>} specifies LSP holding priority
    \item \ttt{<bandwidth></bandwidth>} specifies the requested BW.
    \item \ttt{<delay></delay>} specifies the requested delay.
    \item \ttt{<route></route>} specifies the explicit route. This is a comma-separated
      list of IP-address, hop-type pairs (also separated by comma).
      A hop type has a value of 1 if the hop is a loose hop and 0 otherwise.
  \end{itemize}
\end{itemize}

The following presents an example file:

\begin{verbatim}
<?xml version="1.0"?>
<!-- Example of traffic control file -->
<traffic>
   <conn>
       <src>10.0.0.1</src>
       <dest>10.0.1.2</dest>
       <setupPri>7</setupPri>
       <holdingPri>7</holdingPri>
       <bandwidth>400</bandwidth>
       <delay>5</delay>
   </conn>
   <conn>
       <src>11.0.0.1</src>
       <dest>11.0.1.2</dest>
       <setupPri>7</setupPri>
       <holdingPri>7</holdingPri>
       <bandwidth>100</bandwidth>
       <delay>5</delay>
   </conn>
</traffic>
\end{verbatim}

An example of using RSVP-TE as signaling protocol can be found in
ExplicitRouting folder distributed with the simulation. In this
example, a network similar to the network in LDP-MPLS example is
setup. Instead of using LDP, "signaling" parameter is set to 2 (value
of RSVP-TE handler). The following xml file is used for traffic
control. Note the explicit routes specified in the second connection.
It indicates that the route is a strict one since the values of every
hop types are 0. The route defined is 10.0.0.1 -> 1.0.0.1 ->
10.0.0.3 -> 1.0.0.4 -> 10.0.0.5 -> 10.0.1.2.

\begin{verbatim}
<?xml version="1.0"?>
<!-- Example of traffic control file -->
<traffic>
    <conn>
        <src>10.0.0.1</src>
        <dest>10.0.1.2</dest>
        <setupPri>7</setupPri>
        <holdingPri>7</holdingPri>
        <bandwidth>0</bandwidth>
        <delay>0</delay>
        <ER>false</ER>
    </conn>
    <conn>
        <src>11.0.0.1</src>
        <dest>11.0.1.2</dest>
        <setupPri>7</setupPri>
        <holdingPri>7</holdingPri>
        <bandwidth>0</bandwidth>
        <delay>0</delay>
        <ER>true</ER>
        <route>1.0.0.1,0,1.0.0.3,0,1.0.0.4,0,1.0.0.5,0,10.0.1.2,0</route>
    </conn>
</traffic>
\end{verbatim}

%%% Local Variables:
%%% mode: latex
%%% TeX-master: "usman"
%%% End:


