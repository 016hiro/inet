\chapter{Network Emulation}
\label{cha:emulation}

\section{Motivation}
\label{sec:emulation:motivation}

There are several projects that may benefit from the network emulation
capabilities of INET, that is, from the ability to mix simulated components
with real networks.

Some example scenarios:

\begin{itemize}
  \item Run a simulated component, such as an app or a routing protocol,
    on nodes of an actual ad-hoc network. This setup would allow testing
    the component's behavior under real-life conditions.
  \item Test the interoperability of a simulated protocol with its real-world
    counterparts. Several setups are possible: simulated node in a real network;
    a simulated subnet in real network; real-world node in simulated network; etc.
  \item As a means of implementing hybrid simulation. The real network
    (or a single host OS) may contain several network emulator devices
    or simulations running in emulation mode. Such a setup provides a relatively
    easy way for connecting heterogenous simulators/emulators with each
    other, sparing the need for HLA or a custom interoperability solution.
\end{itemize}


\section{Overview}
\label{sec:emulation:overview}

To act as a network emulator, the simulation must run in real time,
and must be able to communicate with the real world.

% [Background] To understand how it works, first look at how
% a normal simulation is run. There is a scheduler which always
% returns the earliest event from the FES, and the simulation processes
% these events in as fast succession as possible.
%
% For real-time simulation, this scheduler is replaced with one
% augmented with wait calls (e.g. usleep()) that synchronize the
% simulation time to the system clock.
%
% For emulation, the real-time scheduler is augmented with code
% that captures packets from real network devices, and inserts
% them into the simulation.
%
% INET contains an emulation scheduler, which uses the pcap library
% to capture packets, and raw sockets to send packets to a real network device.
%

This is achieved with two components in INET:

\begin{itemize}
  \item \nedtype{ExtInterface} is an INET network interface that represents
    a real interface (an interface of the host OS) in the simulation.
    Packets sent to an \nedtype{ExtInterface} will be sent out on the
    host OS interface, and packets received by the host OS interface
    (or rather, the appropriate subset of them) will appear in the
    simulation as if received on an \nedtype{ExtInterface}. The code
    uses the pcap library for capturing packets, and raw sockets for sending.
 \item \cppclass{RealTimeScheduler}, a socket-aware real-time scheduler class.
\end{itemize}

\begin{note}
It is probably needless to say, but the simulation must be fast enough
to be able to keep up with real time. That is, its relative speed compared
to real time (the simsec/sec value) must be $>>1$.  (Under Qtenv, this
can usually only be achieved in Express mode.)
\end{note}

The simulation is run under Qtenv,

\section{Preparation}
\label{sec:emulation:preparation}

There are a few things that need to be arranged before you can successfully
run simulations in network emulation mode.

First, network emulation is a separate \textit{project feature} that needs to
be enabled before it can be used. (Project features can be reviewed and changed
in the \emph{Project | Project Features...} dialog in the IDE.)

The network emulation code makes use of the pcap library, and therefore
it must be available on your system. On Ubuntu, for example, pcap can be
installed with the following command:

\begin{verbatim}
$ sudo apt install libpcap-dev
\end{verbatim}

Also, when running a simulation, make sure you have the necessary permissions.
Sending uses raw sockets (type \ttt{SOCK\_RAW}), which, on many systems,
is only allowed for processes that have root (administrator) privileges.


\section{Configuring}
\label{sec:emulation:configuring}

INET nodes such as \nedtype{StandardHost} and \nedtype{Router}
can be configured to have \nedtype{ExtInterface}s.
The simulation may contain several nodes with external interfaces,
and one node may also have several external interfaces.

A network node can be configured to have an external interface
in the following way:

\begin{inifile}
**.host1.numExtInterfaces = 1
\end{inifile}

Also, the simulation must be configured to run under control the of the
appropriate real-time scheduler class:

\begin{inifile}
scheduler-class = "inet::RealTimeScheduler"
\end{inifile}

\nedtype{ExtInterface} has two important parameters which need to be
configured. The \fpar{device} parameter should be set to the name of the real
interface on the host OS, and \fpar{filterString} should contain a packet
filter expression that selects which packets captured on the real interface
should  be relayed into the simulation via this \nedtype{ExtInterface}.
(\fpar{filterString} is simply passed to the pcap library, so it should
follow the \textit{tcpdump} filter expressions syntax that pcap understands.)

An example configuration:

\begin{inifile}
**.numExtInterfaces = 1
**.ext[0].ext.filterString = "(sctp or icmp) and ip dst host 10.1.1.1"
**.ext[0].ext.device = "eth0" # or "en0" on macOS, or something
**.ext[0].ext.mtu = 1500B
\end{inifile}

The filter string \ttt{"(sctp or icmp) and ip dst host 10.1.1.1"} means
that the protocol must be SCTP or ICMP, and the destination host must be
10.1.1.1.

\begin{note}
Why is filtering of incoming packets done at packet capture (in pcap),
and not in \nedtype{ExtInterface}? The reason is performance: it costs
much fewer CPU cycles to discard unnecessary packets right where
they come in, and not send them up into the simulation for the
same decision. And, given that the simulation needs to keep up with
real time, saving CPU cycles is important.
\end{note}

Let us examine the paths outgoing and incoming packets take, and the
necessary configuration requirements to make them work. We assume IPv4
as network layer protocol, but the picture does not change much with
other protocols. We assume the external interface is named \ttt{ext[0]}.

\subsection*{Outgoing path}

The network layer of the simulated node routes datagrams to its
\ttt{ext[0]} external interface.

For that to happen, the routing table needs to contain an entry
where the interface is set to \ttt{ext[0]}. Such entries are
not created automatically, one needs to add them to the routing
table explicitly, e.g. by using an \nedtype{Ipv4NetworkConfigurator}
and an appropriate XML file.

Another point is that if the packet comes from a local app (and from
another simulated node), it needs to have a source IP address assigned.
There are two ways for that to happen. If the sending app specified
a source IP address, that will be used. Otherwise, the IP address
of the \ttt{ext[0]} interface will be used, but for that, the interface
needs to have an IP address at all.

Once in \ttt{ext[0]}, the datagram is serialized.
Serialization is a built-in feature of INET packets. (Packets, or rather,
packet chunks have multiple alternative representations, i.e. C++ object
and serialized form, and conversion between them is transparent.)

The result of serialization is a byte string, which is written into
a raw socket with a \ttt{sendto} system call.

% TODO this was true in 4.0 prereleases, hopefully fixed in final 4.0:
% The host OS will again route the packet, and send it out on the
% output interface determined by the packet's destination IP address.
% (Note that this may not be the same as the interface specified
% in \ttt{ext[0]}'s \fpar{device} parameter. Unfortunately,
% in the current implementation one raw socket is shared by
% all \nedtype{ExtInterface} instances which may have different
% \fpar{device} settings, so the socket cannot be bound to the
% real interface associated with originator \nedtype{ExtInterface}.)

The packet will then travel normally in the real network to the
destination address.

\subsection*{Incoming path}

First of all, packets intended to be received by the simulation
need to find their way to the correct  interface of the host that
runs the simulation. For that, IP addresses of simulated hosts
must be routable in the real network, and routed to the captured
interface of the host OS. (On Linux, for example, this can be achieved
by adding static routes with the \ttt{ip route add <prefix> via
<host>} command.)

As packets are received by the interface of the host OS, they
are examined by the pcap library to find out whether they match
the filter expression. If the filter matches, pcap hands the
packet over to the simulation, and after deserialization
it pops out of \ttt{ext[0]} and sent up to the network
layer. After that, it is routed to the simulated destination host
in the normal way.

The pcap filter expression must be crafted so that it matches
the packets destined to simulated hosts, and does not match
any other packet.

Moreover, if the simulation contains several external interfaces
that map to the same real interface, care must be taken so that
filter expressions are disjunct. Otherwise, a packet may be
matched by more than one filter, and then it will be inserted
into the simulation in multiple copies (once for each matching
\nedtype{ExtInterface}.) This is usually not what is wanted.

%%% Local Variables:
%%% mode: latex
%%% TeX-master: "usman"
%%% End:

