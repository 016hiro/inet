\chapter{The Physical Environment}
\label{cha:environment}

\section{Overview}

Todays wireless communication networks are heavily affected by their physical
environment. Mobile networks operate in densely built urban areas, wireless
networks work inside large buildings, low power wireless sensors communicate
in industrial environments, batteries operate in various external conditions,
and so on.

The propagation of radio signals, the movement of communicating agents, battery
exhaustion, etc., depend on the surrounding physical environment. For example,
signals can be absorbed by objects, can pass through objects, can be refracted
by surfaces, can be reflected from surfaces, battery nominal might depend on
external temperature, etc. These effects cannot be ignored in high fidelity
simulations.

In order to help the modeling process the physical environment model is
separated from the rest of the simulation models. It's main goal is to describe
buildings, walls, vegetation, terrain, weather, and other physical conditions
that might have effects on radio signal propagation, movement, batteries, etc.
This separation makes the model reusable by all other simulation models that
depend on these circumstances.

The following sections provide a brief overview of the components of the
physical environment model. 

\section{Physical Objects}

The most important aspect of the physical environment is the objects which are
present in it. For example, simulating an indoor wifi scenario needs to model
walls, floors, ceilings, doors, windows, furniture, etc. These objects are
located in space, most of them have very basic shapes and homogeneous materials.
In short physical objects have the following properties:

\begin{itemize}
  \item \ttt{shape}: describes the 3 dimensional shape of the object
independently of its position and orientation
  \item \ttt{position}: determines where the object is located in the 3
dimensional space
  \item \ttt{orientation}: determines how the object is rotated relative to its
default orientation
  \item \ttt{material}: describes various material specific physical properties  
  \item \ttt{graphical properties}: provides parameters for better visualization
\end{itemize}

The physical objects are also mostly immobile, so in the current model they
cannot change their position or orientation over time.

Since the shape of the physical objects might be quite diverse, the model is
designed to be extensible with new shapes. Concave shapes are not yet supported,
such shapes should be split up into smaller convex parts. The current
implementation provides the following convex shapes:

\begin{itemize}
  \item \cppclass{Sphere}: specified by a radius 
  \item \cppclass{Cuboid}: specified by a length, a width, and a height
  \item \cppclass{Prism}: specified by a 2 dimensional polygon base and a height
  \item \cppclass{Polyhedron}: specified by the convex hull of a set of 3
dimensional vertices 
\end{itemize}

In order to model the physical environment in detail a scenario might contain
several thousands or even more physical objects. Other simulation models might
need to query these objects quite often. For example, when the physical layer
computes obstacle loss for a transmission, it needs to find the obstructing
physical objects for each receiver. This requires computing the intersection
between physical objects and the path traveled by the radio signal. For this
purpose, the physical environment supports storing physical objects in one of
the following efficient cache data structures:

\begin{itemize}
  \item \nedtype{GridObjectCache}: organizes objects into a spatial grid with
a configurable constant cell size, cells contain those objects that intersect
with them
  \item \nedtype{BVHObjectCache}: organizes objects into a tree data structure,
tree leaves contain a configurable number of closely positioned objects
\end{itemize}

The physical environment uses a \nedtype{GridObjectCache} data structure by
default.

\section{Global Physical Properties}

In order to avoid incorrectly positioned physical objects, to limit propagation
and reflection of radio signals, to constrain movement of communicating agents,
to build small and efficient cache data structures, it's necessary to have space
limits. In addition, maximum capacity, internal resistance, self-discharge, etc.,
of battery models might depend on the external temperature. For this purpose,
the  physical environment provides the following global properties:

\begin{itemize}
  \item \ttt{space limits}: global constraints for the 3 dimensional space
  \item \ttt{temperature}: global parameter for temperature dependent models 
\end{itemize}

\section{Initialization}

In order to easily define thousands of physical objects the model needs to use a
flexible and concise representation. For this purpose, the physical environment
uses an XML file format. This file can be used to define physical objects along
with shapes, materials used by them. The following example demonstrates the
syntax of this XML file format:

\begin{verbatim}
<environment>
  <!-- defines a sphere shape -->
  <shape id="1" type="sphere" radius="10"/>
  <!-- defines a cuboid shape -->
  <shape id="2" type="cuboid" size="20 30 40"/>
  <!-- defines a prism (e.g. a cube) shape -->
  <shape id="3" type="prism" height="100"
         points="0 0 100 0 100 100 0 100"/>
  <!-- defines a polyhedron (e.g. a cube) shape -->
  <shape id="4" type="polyhedron"
         points="0 0 0 100 0 0 100 100 0 0 100 0 ..."/>
  <!-- defines a material -->
  <material id="1" resistivity="100"
            relativePermittivity="4.5" relativePermeability="1"/>
  <!-- defines an object using the previously defined shapes and materials -->
  <object position="min 100 200 0" orientation="45 -30 0"
          shape="1" material="1"
          line-color="0 0 0" fill-color="112 128 144" opacity="0.5"/>
  <!-- defines another object similar to the previous one
       using an inline shape and a predefined material -->
  <object position="min 100 200 0" orientation="45 -30 0"
          shape="cuboid 20 30 40" material="concrete"
          line-color="0 0 0" fill-color="112 128 144" tags="Building"/>
</environment>          
\end{verbatim}

For more details on the XML file format please refer to the documentation in the
\nedtype{PhysicalEnvironment} NED file.

\section{Visualization}

Understanding the abstract physical environment model from numerical log output,
or debugging simulation models based on the numerical data structures is
difficult. Therefore the physical environment supports visualizing the physical
objects on the graphical user interface. The visualization is done by drawing
the physical objects on the parent compound module as a separate layer. In order
to help to distinguish among physical objects they have the following graphical
properties:

\begin{itemize}
  \item \ttt{line width}: surface outline 
  \item \ttt{line color}: surface outline
  \item \ttt{fill color}: surface filling 
  \item \ttt{opacity}: surface outline and filling
  \item \ttt{tags}: allows filtering objects on the graphical user interface
\end{itemize}

Although the physical objects are really modeled in 3 dimensions, the
visualization is only 2 dimensional. The projection which converts physical
objects to 2 dimensional shapes can be parameterized with an arbitrary view
angle. The default view angle is the Z axis, in which case the layer shows the
physical objects from above. The view angle can be changed during runtime using
the parameter editor of the graphical user interface.

The projection is also used by other models for their own visualizations. This
approach makes visualizations well integrated with each other. For example, the
mobility models position network nodes, the physical layer draws ongoing
transmissions, the obstacle loss draws intersections between transmissions and
objects, etc., on the canvas according to the projection described above. 

%%% Local Variables:
%%% mode: latex
%%% TeX-master: "usman"
%%% End:

