\ifdraft TODO

\chapter{Developing Protocol and Application Models}
\label{cha:developing-models}

\section{Overview}

This section introduces the most important modeling support features of
INET. These features facilitate the implementation of applications and
communication protocols by providing various commonly used functionality.
Thus modeling support allows rapid implementation of new models by building
on already existing APIs while the implementor can focus on the research
topics. These features differ from the reusable NED modules introduced
earlier, because they are available in the form of C++ APIs.

The easy usage of protocol services is another essential modeling support.
Applications often need to use several different protocol services
simultaneously. In order to spare the applications from using the default
OMNeT++ message passing style between modules, INET provides an easy to use
C++ socket API.

\ifdraft TODO
\section{NED Conventions}

%TODO conventions for NED module definitions

\subsection{The @node Property}

By convention, compound modules that implement network devices (hosts,
routers, switches, access points, base stations, etc.) are marked with the
\ttt{@node} NED property. As node models may themselves be hierarchical, the
\ttt{@node} property is used by protocol implementations and other simple
modules to determine which ancestor compound module represents the physical
network node they live in.

\subsection{The @labels Module Property}

The \ttt{@labels} property can be added to modules and gates, and it
allows the OMNeT++ graphical editor to provide better editing experience.
First we look at \ttt{@labels} as a module property.

\ttt{@labels(node)} has been added to all NED module types that may occur on
network level. When editing a network, the graphical editor will NED types
with \ttt{@labels(node)} to the top of the component palette, allowing the
user to find them easier.

Other labels can also be specified in the \ttt{@labels(...)} property. This
has the effect that if one module with a particular label has already been
added to the compound module being edited, other module types with the same
label are also brought to the top of the palette. For example,
\nedtype{EtherSwitch} is annotated with \ttt{@labels(node,ethernet-node)}.
When you drop an \nedtype{EtherSwitch} into a compound module, that will
bring \nedtype{EtherHost} (which is also tagged with the
\ttt{ethernet-node} label) to the top of the palette, making it easier to
find.

\begin{ned}
module EtherSwitch
{
    parameters:
        @node();
        @labels(node,ethernet-node);
        @display("i=device/switch");
    ...
}
\end{ned}

Module types that are already present in the compound module also appear in
the top part of the palette. The reason is that if you already added a
\nedtype{StandardHost}, for example, then you are likely to add more of the
same kind. Gate labels (see next section) also affect palette order: modules
which can be connected to modules already added to the compound module
will also be listed at the top of the palette. The final ordering is the
result of a scoring algorithm.


\subsection{The @labels Gate Property}

Gates can also be labelled with \ttt{@labels()}; the purpose is to make it easier
to connect modules in the editor. If you connect two modules in the editor,
the gate selection menu will list gate pairs that have a label in common.

\ifdraft TODO
screenshot
\fi

For example, when connecting hosts and routers, the editor will offer connecting
Ethernet gates with Ethernet gates, and PPP gates with PPP gates. This is the
result of gate labelling like this:

\begin{ned}
module StandardHost
{
    ...
    gates:
        inout pppg[] @labels(PPPFrame-conn);
        inout ethg[] @labels(EtherFrame-conn);
    ...
}
\end{ned}

Guidelines for choosing gate label names: For gates of modules that
implement protocols, use the C++ class name of the packet or acompanying
control info (see later) associated with the gate, whichever applies;
append \ttt{/up} or \ttt{/down} to the name of the control info class. For
gates of network nodes, use the class names of packets (frames) that travel
on the corresponding link, with the \ttt{-conn} suffix. The suffix prevents
protocol-level modules to be promoted in the graphical editor palette when
a network is edited.

Examples:

\begin{ned}
simple TCP like ITCP
{
    ...
    gates:
        input appIn[] @labels(TCPCommand/down);
        output appOut[] @labels(TCPCommand/up);
        input ipIn @labels(TCPSegment,IPv4ControlInfo/up,IPControlInfo/up);
        output ipOut @labels(TCPSegment,IPv4ControlInfo/down,IPv6ControlInfo/up);
}
\end{ned}


\begin{ned}
simple PPP
{
    ...
    gates:
        input netwIn;
        output netwOut;
        inout phys @labels(PPPFrame);
}
\end{ned}
\fi

\section{Module Initialization}

%TODO explain how INET supports multi-stage interdependent module initialization

\cppsnippet{ModuleInitializationExample}{Module initialization example}


\section{Addresses}

\subsection{Address Types}

The INET Framework uses a number of C++ classes to represent various
addresses in the network. These classes support initialization and
assignment from binary and string representation of the address, and
accessing the address in both forms. Storage is in binary form, and they also
support the "unspecified" special value (and the \ffunc{isUnspecified()}
method) that usually corresponds to the all-zeros address.

\begin{itemize}
  \item \cppclass{MacAddress} represents a 48-bit IEEE 802 MAC address. The
    textual notation it understands and produces is hex string.

  \item \cppclass{Ipv4Address} represents a 32-bit IPv4 address. It can parse
    and produce textual representations in the "dotted decimal" syntax.

  \item \cppclass{Ipv6Address} represents a 128-bit IPv6 address. It can parse
    and produce address strings in the canonical (RFC 3513) syntax.

  \item \cppclass{L3Address} is conceptually a union of a \cppclass{Ipv4Address}
    and \cppclass{Ipv6Address}: an instance stores either an IPv4 address or an
    IPv6 address. \cppclass{L3Address} is mainly used in the transport layer and above
    to abstract away network addresses. It can be assigned from both \cppclass{Ipv4Address}
    and \cppclass{Ipv6Address}, and can also parse string representations of both.
    The \ffunc{getType()}, \ffunc{toIpv4()} and \ffunc{toIpv6()} methods can be used
    to access the value.
\end{itemize}

\ifdraft TODO
\subsection{Resolving Addresses}

explain what kind of addresses INET provides for protocols to use: network
and MAC addresses, related protocols: ARP, DHCP, ND, etc.

address lookup by name

node lookup by MAC address

node lookup by L3 address
\fi

\ifdraft TODO
\section{Starting and Stopping Nodes}

%TODO explain how INET supports network node lifecycle management: startup, shutdown, crash, etc.

\cppsnippet{LifecycleOperationExample}{Lifecycle operation example}
\fi

\fi


%%%% Local Variables:
%%% mode: latex
%%% TeX-master: "usman"
%%% End:

