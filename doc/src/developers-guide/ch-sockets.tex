\chapter{Using Sockets}
\label{cha:sockets}

\section{Overview}
\label{sec:sockets:overview}

The INET Socket API provides special C++ abstractions on top of the standard
OMNeT++ message passing interface for several communication protocols.

Sockets are most often used by applications and routing protocols to acccess the
corresponding protocol services. Sockets are capable of communicating with the
underlying protocol in a bidirectional way. They can assemble and send service
requests and packets, and they can also receive service indications and packets.

Applications can simply call the socket class member functions (e.g.
\ffunc{bind()}, \ffunc{connect()}, \ffunc{send()}, \ffunc{close()}) to create
and configure sockets, and to send and receive packets. They may also use
several different sockets simulatenously.

The following sections first introduce the shared functionality of sockets, and
then list all INET sockets in detail, mostly by shedding light on many common
usages through examples.

\begin{note}
Code fragments in this chapter have been somewhat simplified for brevity. For
example, some \ttt{virtual} modifiers and \ttt{override} qualifiers have been
omitted, and some algorithms have been simplified to ease understanding.
\end{note}

\subsection*{Socket Interfaces}

Although sockets are always implemented as protocol specific C++ classes, INET
also provides C++ socket interfaces. These interfaces allow writing general C++
code which can handle many different kinds of sockets all at once.

For example, the \cppclass{ISocket} interface is implemented by all sockets, and
the \cppclass{INetworkSocket} interface is implemented by all network protocol
sockets.

\subsection*{Identifying Sockets}

All sockets have a socket identifier which is unique within the network node. It
is automatically assigned to the sockets when they are created. The identifier
can accessed with \ffunc{getSocketId()} throughout the lifetime of the socket.

The socket identifier is also passed along in \cppclass{SocketReq} and
\cppclass{SocketInd} packet tags. These tags allow applications and protocols to
identify the socket to which \cppclass{Packet}s, service \cppclass{Request}s,
and service \cppclass{Indication}s belong.

\subsection*{Configuring Sockets}

Since all sockets work with message passing under the hoods, they must be
configured prior to use. In order to send packets and service requests on the
correct gate towards the underlying communication protocol, the output gate must
be configured:

\cppsnippet{SocketConfigureExample}{Socket configure example}

In contrast, incoming messages such as service indications from the underlying
communication protocol can be received on any application gate.

To ease application development, all sockets support storing a user specified
data object pointer. The pointer is accessible with the \ffunc{setUserData()},
\ffunc{getUserData()} member functions.

Another mandatory configuration for all sockets is setting the socket callback
interface. The callback interface is covered in more detail in the following
section.

Other socket specific configuration options are also available, these are
discussed in the section of the corresponding socket.

\subsection*{Callback Interfaces}

To ease centralized message processing, all sockets provide a callback interface
which must be implemented by applications. The callback interface is usually
called \cppclass{ICallback}, and it's defined as an inner class of the socket it
belongs to. These interfaces often contain some generic notification methods
along with several socket specific methods.

For example, the most common callback method is the one which processes incoming
packets:

\cppsnippet{SocketCallbackInterfaceExample}{Socket callback interface example}

\subsection*{Processing Messages}

In general, sockets can process all incoming messages which were sent by the
underlying protocol. The received messages must be processed by the socket where
they belong to.

For example, an application can simply go through each knonwn socket in any
order, and decide which one should process the received message as follows:

\cppsnippet{SocketProcessExample}{Socket process example}

Sockets usually deconstruct the received messages and update their state
accordingly if necessary. They also automatically dispatch received packets and
service indications for further processing to the appropriate functions in the
corresponding \cppclass{ICallback} interface.

\subsection*{Sending Data}

All sockets provide one or more \ffunc{send()} functions which send packets using
the current configuration of the socket. The actual means of packet delivery
depends on the underlying communication protocol, but in general the state of
the socket is expected to affect it.

For example, after the socket is properly configured, the application can start
sending packets without attaching any tags, because the socket takes care of
the necessary technical details:

\cppsnippet{SocketSendExample}{Socket send example}

\subsection*{Receiving Data}

For example, the application may directly implement the \cppclass{ICallback}
interface of the socket and print the received data as follows:

\cppsnippet{SocketReceiveExample}{Socket receive example}

\subsection*{Closing Sockets}

Sockets must be closed before deleting them. Closing a socket allows the
underlying communication protocol to release allocated resources. These
resources are often allocated on the local network node, the remote nework node,
or potentially somewhere else in the network.

For example, a socket for a connection oriented protocol must be closed to
release the allocated resources at the peer:

\cppsnippet{SocketCloseExample}{Socket close example}

\subsection*{Using Multiple Sockets}

If the application needs to manage a large number of sockets, for example in a
server application which handles multiple incoming connections, the generic
\cppclass{SocketMap} class may be useful. This class can manage all kinds of
sockets which implement the \cppclass{ISocket} interface simultaneously.

For example, processing an incoming packet or service indication can be done as
follows:

\cppsnippet{SocketFindExample}{Socket find example}

In order for the \cppclass{SocketMap} to operate properly, sockets must be added
to and removed from it using the \ffunc{addSocket()} and \ffunc{removeSocket()}
methods respectively.

\section{UDP Socket}
\label{sec:sockets:udp-socket}

The \cppclass{UdpSocket} class provides an easy to use C++ interface to send and
receive \protocol{UDP} datagrams. The underlying \protocol{UDP} protocol is
implemented in the \nedtype{Udp} module.

\subsection*{Callback Interface}

Processing packets and indications which are received from the \nedtype{Udp}
module is pretty simple. The incoming message must be processed by the socket
where it belongs as shown in the general section.

The \cppclass{UdpSocket} deconstructs the message and uses the
\cppclass{UdpSocket::ICallback} interface to notify the application about
received data and error indications:

\cppsnippet{UdpSocketCallbackInterface}{UDP socket callback interface}

\subsection*{Configuring Sockets}

For receiving \protocol{UDP} datagrams on a socket, it must be bound to an
address and a port. Both the address and port is optional. If the address is
unspecified, than all \protocol{UDP} datagrams with any destination address are
received. If the port is -1, then an unused port is selected automatically by
the \nedtype{Udp} module. The address and port pair must be unique within the
same network node.

Here is how to bind to a specific local address and port to receive
\protocol{UDP} datagrams:

\cppsnippet{UdpSocketBindExample}{UDP socket bind example}

For only receiving \protocol{UDP} datagrams from a specific remote address/port,
the socket can be connected to the desired remote address/port:

\cppsnippet{UdpSocketConnectExample}{UDP socket connect example}

There are several other socket options (e.g. receiving broadcasts, managing
multicast groups, setting type of service) which can also be configured using
the \cppclass{UdpSocket} class:

\cppsnippet{UdpSocketConfigureExample}{UDP socket configuration example}

\subsection*{Sending Data}

After the socket has been configured, applications can send datagrams to a
remote address and port via a simple function call:

\cppsnippet{UdpSocketSendToExample}{UDP socket sendTo example}

If the application wants to send several datagrams, it can optionally connect to
the destination.

The \protocol{UDP} protocol is in fact connectionless, so when the \nedtype{Udp}
module receives the connect request, it simply remembers the remote address and
port, and use it as default destination for later sends.

\cppsnippet{UdpSocketSendExample}{UDP socket send example}

The application can call connect several times on the same socket.

\subsection*{Receiving Data}

For example, the application may directly implement the
\cppclass{UdpSocket::ICallback} interface and print the received data as
follows:

\cppsnippet{UdpSocketReceiveExample}{UDP socket receive example}

\section{TCP Socket}
\label{sec:sockets:tcp-socket}

The \cppclass{TcpSocket} class provides an easy to use C++ interface to manage
\protocol{TCP} connections, and to send and receive data. The underlying
\protocol{TCP} protocol is implemented in the \nedtype{Tcp}, \nedtype{TcpLwip},
and \nedtype{TcpNsc} modules.

\subsection*{Callback Interface}

Messages received from the various \nedtype{Tcp} modules can be processed by the
\cppclass{TcpSocket} where they belong to. The \cppclass{TcpSocket} deconstructs
the message and uses the \cppclass{TcpSocket::ICallback} interface to notify the
application about the received data or service indication:

\cppsnippet{TcpSocketCallbackInterface}{TCP socket callback interface}

\subsection*{Configuring Connections}

The \nedtype{Tcp} module supports several \protocol{TCP} different congestion
algorithms, which can also be configured using the \cppclass{TcpSocket}:

\cppsnippet{TcpSocketConfigureExample}{TCP socket configure example}

Upon setting the individual parameters, the socket immediately sends sevice
requests to the underlying \nedtype{Tcp} protocol module.

\subsection*{Setting up Connections}

Since \protocol{TCP} is a connection oriented protocol, a connection must be
established before applications can exchange data. On the one side, the
application listens at a local address and port for incoming \protocol{TCP}
connections:

\cppsnippet{TcpSocketListenExample}{TCP socket listen example}

On the other side, the application connects to a remote address and port to
establish a new connection:

\cppsnippet{TcpSocketConnectExample}{TCP socket connect example}

\subsection*{Accepting Connections}

The \nedtype{Tcp} module automatically notifies the \cppclass{TcpSocket} about
incoming connections. The socket in turn notifies the application using the
\ffunc{ICallback::socketAvailable} method of the callback interface. Finally,
incoming \protocol{TCP} connections must be accepted by the application before
they can be used:

\cppsnippet{TcpSocketAcceptExample}{TCP socket accept example}

After the connection is accepted, the \nedtype{Tcp} module notifies the
application about the socket being established and ready to be used.

\subsection*{Sending Data}

After the connection has been established, applications can send data to the
remote application via a simple function call:

\cppsnippet{TcpSocketSendExample}{TCP socket send example}

Packet data is enqueued by the local \nedtype{Tcp} module and transmitted
over time according to the protocol logic.

\subsection*{Receiving Data}

Receiving data is as simple as implementing the corresponding method of the
\cppclass{TcpSocket::ICallback} interface. One caveat is that packet data may
arrive in different chunk sizes (but the same order) than they were sent due to
the nature of \protocol{TCP} protocol.

For example, the application may directly implement the
\cppclass{TcpSocket::ICallback} interface and print the received data as
follows:

\cppsnippet{TcpSocketReceiveExample}{TCP socket receive example}

\section{SCTP Socket}
\label{sec:sockets:sctp-socket}

The \cppclass{SctpSocket} class provides an easy to use C++ interface to manage
\protocol{SCTP} connections, and to send and receive data. The underlying
\protocol{SCTP} protocol is implemented in the \nedtype{Sctp} module.

\subsection*{Callback Interface}

Messages received from the \nedtype{Sctp} module can be processed by the
\cppclass{SctpSocket} where they belong to. The \cppclass{SctpSocket}
deconstructs the message and uses the \cppclass{SctpSocket::ICallback} interface
to notify the application about the received data or service indication:

\cppsnippet{SctpSocketCallbackInterface}{SCTP socket callback interface}

\subsection*{Configuring Connections}

The \cppclass{SctpSocket} class supports setting several \protocol{SCTP}
specific connection parameters directly:

\cppsnippet{SctpSocketConfigureExample}{SCTP socket configure example}

Upon setting the individual parameters, the socket immediately sends sevice
requests to the underlying \nedtype{Sctp} protocol module.

\subsection*{Setting up Connections}

Since \protocol{SCTP} is a connection oriented protocol, a connection must be
established before applications can exchange data. On the one side, the
application listens at a local address and port for incoming \protocol{SCTP}
connections:

\cppsnippet{SctpSocketListenExample}{SCTP socket listen example}

On the other side, the application connects to a remote address and port to
establish a new connection:

\cppsnippet{SctpSocketConnectExample}{SCTP socket connect example}

\subsection*{Accepting Connections}

The \nedtype{Sctp} module automatically notifies the \cppclass{SctpSocket} about
incoming connections. The socket in turn notifies the application using the
\ffunc{ICallback::socketAvailable} method of the callback interface. Finally,
incoming \protocol{SCTP} connections must be accepted by the application before
they can be used:

\cppsnippet{SctpSocketAcceptExample}{SCTP socket accept example}

\subsection*{Sending Data}

After the connection has been established, applications can send data to the
remote applica- tion via a simple function call:

\cppsnippet{SctpSocketSendExample}{SCTP socket send example}

Packet data is enqueued by the local \nedtype{Sctp} module and transmitted over
time according to the protocol logic.

\subsection*{Receiving Data}

Receiving data is as simple as implementing the corresponding method of the
\cppclass{SctpSocket::ICallback} interface. One caveat is that packet data may
arrive in different chunk sizes (but the same order) than they were sent due to
the nature of \protocol{SCTP} protocol.

For example, the application may directly implement the
\cppclass{SctpSocket::ICallback} interface and print the received data as
follows:

\cppsnippet{SctpSocketReceiveExample}{SCTP socket receive example}

\section{IPv4 Socket}
\label{sec:sockets:ipv4-socket}

The \cppclass{Ipv4Socket} class provides an easy to use C++ interface to send
and receive \protocol{IPv4} datagrams. The underlying \protocol{IPv4} protocol
is implemented in the \nedtype{Ipv4} module.

\subsection*{Callback Interface}

Messages received from the \nedtype{Ipv4} module must be processed by the socket
where they belong as shown in the general section. The \cppclass{Ipv4Socket}
deconstructs the message and uses the \cppclass{Ipv4Socket::ICallback} interface
to notify the application about the received data:

\cppsnippet{Ipv4SocketCallbackInterface}{IPv4 socket callback interface}

\subsection*{Configuring Sockets}

In order to only receive \protocol{IPv4} datagrams which are sent to a specific
local address or contain a specific protocol, the socket can be bound to the
desired local address or protocol.

For example, the following code fragment shows how the INET \nedtype{PingApp}
binds to the \protocol{ICMPv4} protocol to receive all incoming
\protocol{ICMPv4} Echo Reply messages:

\cppsnippet{Ipv4SocketBindExample}{IPv4 socket bind example}

For only receiving \protocol{IPv4} datagrams from a specific remote address, the
socket can be connected to the desired remote address:

\cppsnippet{Ipv4SocketConnectExample}{IPv4 socket connect example}

\subsection*{Sending Data}

After the socket has been configured, applications can immediately start sending
\protocol{IPv4} datagrams to a remote address via a simple function call:

\cppsnippet{Ipv4SocketSendToExample}{IPv4 socket send to example}

If the application wants to send several \protocol{IPv4} datagrams to the same
destination address, it can optionally connect to the destination:

\cppsnippet{Ipv4SocketSendExample}{IPv4 socket send example}

The \protocol{IPv4} protocol is in fact connectionless, so when the
\nedtype{Ipv4} module receives the connect request, it simply remembers the
remote address, and uses it as the default destination address for later sends.

The application can call \ffunc{connect()} several times on the same socket.

\subsection*{Receiving Data}

For example, the application may directly implement the
\cppclass{Ipv4Socket::ICallback} interface and print the received data as
follows:

\cppsnippet{Ipv4SocketReceiveExample}{IPv4 socket receive example}

\section{IPv6 Socket}
\label{sec:sockets:ipv6-socket}

The \cppclass{Ipv6Socket} class provides an easy to use C++ interface to send
and receive \protocol{IPv6} datagrams. The underlying \protocol{IPv6} protocol
is implemented in the \nedtype{Ipv6} module.

\subsection*{Callback Interface}

Messages received from the \nedtype{Ipv6} module must be processed by the socket
where they belong as shown in the general section. The \cppclass{Ipv6Socket}
deconstructs the message and uses the \cppclass{Ipv6Socket::ICallback} interface
to notify the application about the received data:

\cppsnippet{Ipv6SocketCallbackInterface}{IPv6 socket callback interface}

\subsection*{Configuring Sockets}

In order to only receive \protocol{IPv6} datagrams which are sent to a specific
local address or contain a specific protocol, the socket can be bound to the
desired local address or protocol.

For example, the following code fragment shows how the INET \nedtype{PingApp}
binds to the \protocol{ICMPv6} protocol to receive all incoming
\protocol{ICMPv6} Echo Reply messages:

\cppsnippet{Ipv6SocketBindExample}{IPv6 socket bind example}

For only receiving \protocol{IPv6} datagrams from a specific remote address, the
socket can be connected to the desired remote address:

\cppsnippet{Ipv6SocketConnectExample}{IPv6 socket connect example}

\subsection*{Sending Data}

After the socket has been configured, applications can immediately start sending
\protocol{IPv6} datagrams to a remote address via a simple function call:

\cppsnippet{Ipv6SocketSendAtExample}{IPv6 socket send at example}

If the application wants to send several \protocol{IPv6} datagrams to the same
destination address, it can optionally connect to the destination:

\cppsnippet{Ipv6SocketSendExample}{IPv6 socket send example}

The \protocol{IPv6} protocol is in fact connectionless, so when the
\nedtype{Ipv6} module receives the connect request, it simply remembers the
remote address, and uses it as the default destination address for later sends.

The application can call \ffunc{connect()} several times on the same socket.

\subsection*{Receiving Data}

For example, the application may directly implement the
\cppclass{Ipv6Socket::ICallback} interface and print the received data as
follows:

\cppsnippet{Ipv6SocketReceiveExample}{IPv6 socket receive example}

\section{L3 Socket}
\label{sec:sockets:l3-socket}

The \cppclass{L3Socket} class provides an easy to use C++ interface to send and
receive datagrams using the conceptual network protocols. The underlying network
protocols are implemented in the \nedtype{NextHopForwarding},
\nedtype{Flooding}, \nedtype{ProbabilisticBroadcast}, and
\nedtype{AdaptiveProbabilisticBroadcast} modules.

\subsection*{Callback Interface}

Messages received from the network protocol module must be processed by the
associated socket where as shown in the general section. The \cppclass{L3Socket}
deconstructs the message and uses the \cppclass{L3Socket::ICallback} interface
to notify the application about the received data:

\cppsnippet{L3SocketCallbackInterface}{L3 socket callback interface}

\subsection*{Configuring Sockets}

Since the \cppclass{L3Socket} class is network protocol agnostic, it must be
configured to connect to a desired network protocol:

\cppsnippet{L3SocketProtocolExample}{L3 socket protocol example}

In order to only receive datagrams which are sent to a specific local address or
contain a specific protocol, the socket can be bound to the desired local
address or protocol. The conceptual network protocols can work with the
\cppclass{ModuleIdAddress} class which contains a \fvar{moduleId} of the desired
network interface.

For example, the following code fragment shows how the INET \nedtype{PingApp}
binds to the \protocol{Echo} protocol to receive all incoming \protocol{Echo}
Reply messages:

\cppsnippet{L3SocketBindExample}{L3 socket bind example}

For only receiving datagrams from a specific remote address, the socket can be
connected to the desired remote address:

\cppsnippet{L3SocketConnectExample}{L3 socket connect example}

\subsection*{Sending Data}

After the socket has been configured, applications can immediately start sending
datagrams to a remote address via a simple function call:

\cppsnippet{L3SocketSendToExample}{L3 socket send to example}

If the application wants to send several datagrams to the same destination
address, it can optionally connect to the destination:

\cppsnippet{L3SocketSendExample}{L3 socket send example}

The network protocols are in fact connectionless, so when the protocol module
receives the connect request, it simply remembers the remote address, and uses
it as the default destination address for later sends.

The application can call \ffunc{connect()} several times on the same socket.

\subsection*{Receiving Data}

For example, the application may directly implement the
\cppclass{L3Socket::ICallback} interface and print the received data as follows:

\cppsnippet{L3SocketReceiveExample}{L3 socket receive example}

\section{TUN Socket}
\label{sec:sockets:tun-socket}

The \cppclass{TunSocket} class provides an easy to use C++ interface to send and
receive datagrams using a \protocol{TUN} interface. The underlying
\protocol{TUN} interface is implemented in the \nedtype{Tun} module.

A \protocol{TUN} interface is basically a virtual network interface which is
usually connected to an application (from the outside) instead of other network
devices. It can be used for many networking tasks such as tunneling, or virtual
private networking.

\subsection*{Callback Interface}

Messages received from the \nedtype{Tunq} module must be processed by the socket
where they belong as shown in the general section. The \cppclass{TunSocket}
deconstructs the message and uses the \cppclass{TunSocket::ICallback} interface
to notify the application about the received data:

\cppsnippet{TunSocketCallbackInterface}{TUN socket callback interface}

\subsection*{Configuring Sockets}

A \cppclass{TunSocket} must be associated with a \protocol{TUN} interface before
it can be used:

\cppsnippet{TunSocketOpenExample}{TUN socket open example}

\subsection*{Sending Packets}

As soon as the \cppclass{TunSocket} is associated with a \protocol{TUN}
interface, applications can immediately start sending datagrams via a simple
function call:

\cppsnippet{TunSocketSendExample}{TUN socket send example}

When the application sends a datagram to a \cppclass{TunSocket}, the packet
appears for the protocol stack within the network node as if the packet were
received from the network.

\subsection*{Receiving Packets}

Messages received from the \protocol{TUN} interface must be processed by the
corresponding \cppclass{TunSocket}. The \cppclass{TunSocket} deconstructs the
message and uses the \cppclass{TunSocket::ICallback} interface to notify the
application about the received data:

\cppsnippet{TunSocketReceiveExample}{TUN socket receive example}

When the protocol stack within the network node sends a datagram to a
\protocol{TUN} interface, the packet appears for the application which uses a
\cppclass{TunSocket} as if the packet were sent to the network.



\ifdraft TODO
\section{IEEE 802.2 Socket}

\section{Ethernet Socket}

\section{IEEE 802.11 Socket}
\fi
