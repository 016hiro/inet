\chapter{Introduction}
\label{cha:introduction}

\section{What is INET Framework}
\label{sec:introduction:what-is-inet}

INET Framework is an open-source model library for the OMNeT++ simulation
environment. It provides protocols, agents and other models for researchers and
students working with communication networks. INET is especially useful when
designing and validating new protocols, or exploring new or exotic scenarios.

INET supports a wide class of communication networks, including wired, wireless,
mobile, ad hoc and sensor networks.  It contains models for the Internet stack
(TCP, UDP, IPv4, IPv6, OSPF, BGP, etc.), link layer protocols (Ethernet, PPP,
IEEE 802.11, various sensor MAC protocols, etc), refined support for the
wireless physical layer, MANET routing protocols, DiffServ, MPLS with LDP and
RSVP-TE signalling, several application models, and many other protocols and
components. It also provides support for node mobility, advanced visualization,
network emulation and more.

Several other simulation frameworks take INET as a base, and extend it into
specific directions, such as vehicular networks, overlay/peer-to-peer networks,
or LTE.

\section{Scope of this Manual}
\label{sec:introduction:scope-of-this-manual}

This manual is written for developers who intend to extend INET with new
components, written in C++. This manual is accompanied by the INET Reference,
which is generated from NED and MSG files using OMNeT++'s documentation
generator, and the documentation of the underlying C++ classes, generated from
the source files using Doxygen. A working knowledge of OMNeT++ and the C++
language is assumed.


