\chapter{Getting Started}
\label{cha:gettingstarted}

\section{Introduction}
\label{cha:gettingstarted:introduction}

where to put the source files: you can copy and modify the INET framework (fork it)
in the hope that you'll contribute back the changes; or you can develop in
a separate project (create new project in the IDE; mark INET as referenced project)

\section{Contributing to INET}
\label{cha:gettingstarted:contributing-to-inet}

Workflow:

Fork on Github.

Check out the INET project from GitHub, and import it into the OMNeT++ IDE.

Develop.

Submit pull requests.


\section{Setting Up a New INET-Based Project}
\label{cha:gettingstarted:setting-up-inet-based}

Create new project in the IDE.

NED and source files in the same folder; examples under examples/; etc.

Set INET as referenced project.

Set up version control (git, GitHub).

Develop.




