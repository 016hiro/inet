The LargeNet Model

NOTE: this file contains the desctiption of the LargeNet model, from
examples/ethernet/lans, which could be turned into an Ethernet showcase.


The \nedtype{LargeNet} model demonstrates how one can put together models of large
LANs with little effort, making use of MAC auto-configuration.

\nedtype{LargeNet} models a large Ethernet campus backbone. As configured in the
default omnetpp.ini, it contains altogether about 8000 computers
and 900 switches and hubs. This results in about 165MB process size
on my (32-bit) linux box when I run the simulation.
The model mixes all kinds of Ethernet technology: Gigabit Ethernet,
100Mb full duplex, 100Mb half duplex, 10Mb UTP, 10Mb bus ("thin Ethernet"),
switched hubs, repeating hubs.

The topology is in \nedtype{LargeNet}.ned, and it looks like this: there's chain
of n=15 large "backbone" switches (switchBB[]) as well as four more
large switches (switchA, switchB, switchC, switchD) connected to
somewhere the middle of the backbone (switchBB[4]). These 15+4 switches
make up the backbone; the n=15 number is configurable in omnetpp.ini.

Then there're several smaller LANs hanging off each backbone switch.
There're three types of LANs: small, medium and large (represented by
compound module types \nedtype{SmallLAN}, \nedtype{MediumLAN}, \nedtype{LargeLAN}). A small LAN
consists of a few computers on a hub (100Mb half duplex); a medium
LAN consists of a smaller switch with a hub on one of its port
(and computers on both); the large one also has a switch and a hub,
plus an Ethernet bus hanging of one port of the hub (there's still hubs
around with one BNC connector besides the UTP ones).
By default there're 5..15 LANs of each type hanging off each backbone
switch. (These numbers are also omnetpp.ini parameters like the length
of the backbone.)

The application model which generates load on the simulated LAN is
simple yet powerful. It can be used as a rough model for any
request-response based protocol such as SMB/CIFS (the Windows file
sharing protocol), HTTP, or a database client-server protocol.

Every computer runs a client application (\nedtype{EtherAppCli}) which connects
to one of the servers. There's one server attached to switches A, B,
C and D each: serverA, serverB, serverC and serverD -- server selection
is configured in omnetpp.ini). The servers run \nedtype{EtherAppSrv}.
Clients periodically send a request to the server, and the request
packet contains how many bytes the client wants the server to send back
(this can mean one or more Ethernet frames, depending on the byte count).
 Currently the request and reply lengths are configured in omnetpp.ini
as intuniform(50,1400) and truncnormal(5000,5000).

The volume of the traffic can most easily be controlled with the
time period between sending requests; this is currently
set in omnetpp.ini to exponential(0.50) (that is, average 2
requests per second). This already causes frames to be dropped
in some of the backbone switches, so the network is a bit
overloaded with the current settings.

The model generates extensive statistics. All MACs (and most other
modules too) write statistics into omnetpp.sca at the end
of the simulation: number of frames sent, received, dropped, etc.
These are only basic statistics, however it still makes the
scalar file to be several ten megabytes in size. You can use
the analysis tools provided with OMNeT++ to visualized the data
in this file. (If the file size is too big, writing statistics
can be disabled, by putting **.record-scalar=false in the ini file.)
The model can also record output vectors, but this is currently
disabled in omnetpp.ini because the generated file can easily reach
gigabyte sizes.

